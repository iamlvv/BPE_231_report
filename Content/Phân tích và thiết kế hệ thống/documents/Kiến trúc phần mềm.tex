\subsection{Kiến trúc phần mềm}
Với những nghiệp vụ được đề cập ở trên, chúng tôi quyết định lựa chọn hướng tiếp cận của kiến trúc đa lớp cho hệ thống của chúng tôi. Kiến trúc đa lớp là một trong những kiến trúc phổ biến trong các loại kiến trúc phần mềm. Kiến trúc này ra đời nhằm phân chia các thành phần trong hệ thống, các thành phần cùng chức năng sẽ được nhóm lại với nhau và phân chia công việc cho từng nhóm để dữ liệu không bị chồng chéo và lộn xộn. Kiến trúc này phát huy hiệu quả nhất ở các hệ thống từ nhỏ đến lớn, giúp cho việc quản lý code và xử lý dữ liệu lỗi dễ dàng hơn.
\par
Mặc dù không có quy định cụ thể về số lượng hay kiểu của các lớp, ở những hệ thống phức tạp hơn có thể có nhiều lớp hơn, đa số các kiến trúc đa lớp gồm có ba lớp chuẩn (3-tier): \emph{Presentation Layer}, \emph{Business Layer}, \emph{Data Layer}. \emph{Presentation layer} có nhiệm vụ chính giao tiếp với người dùng, gồm các thành phần giao diện và thực hiện các công việc như nhập liệu, hiển thị dữ liệu, kiểm tra tính đúng đắn của dữ liệu trước khi gọi lớp \emph{Business layer}. \emph{Business layer} phân ra thành hai nhiệm vụ: thứ nhất, đây là nơi đáp ứng các yêu cầu thao tác dữ liệu của \emph{Presentation layer}, xử lý chính nguồn dữ liệu từ \emph{Presentation layer} trước khi truyền xuống \emph{Data layer} và lưu xuống hệ quản trị cơ sở dữ liệu. \emph{Data layer} có chức năng giao tiếp với hệ quản trị CSDL như thực hiện các công việc liên quan đến lưu trữ và truy vấn dữ liệu ( tìm kiếm, thêm, xóa, sửa,…).
\par
Kiến trúc này có một số ưu điểm như sau:
\begin{itemize}
    \item Việc phân chia thành từng lớp giúp cho code được tường minh hơn. Nhờ vào việc chia ra từng lớp đảm nhận các chức năng khác nhau và riêng biệt như giao diện, xử lý, truy vấn thay vì để tất cả lại một chỗ, nhằm giảm sự kết dính.
    \item Dễ bảo trì khi được phân chia, thì một thành phần của hệ thống sẽ dễ thay đổi. Việc thay đổi này có thể được cô lập trong 1 lớp, hoặc ảnh hưởng đến lớp gần nhất mà không ảnh hưởng đến cả chương trình.
    \item Dễ phát triển, tái sử dụng: khi chúng ta muốn thêm một chức năng nào đó thì việc lập trình theo một mô hình sẽ dễ dàng hơn vì chúng ta đã có chuẩn để tuân theo. Và việc sử dụng lại  khi có sự thay đổi giữa hai môi trường thì chỉ việc thay đổi lại \emph{Presentation layer}.
\end{itemize}

\par
Ngoài kiến trúc đa lớp này, chúng tôi còn tìm hiểu thêm một loại kiến trúc cũng rất phổ biến hiện nay, đó là Microservices. Theo kiến trúc này, một ứng dụng được chia thành một bộ các microservice, mỗi microservice thực chất là một service có thể được triển khai và chạy độc lập. Chúng tách biệt về mặt mã nguồn, về hoạt động và dữ liệu. Mỗi microservice có nơi chứa dữ liệu của riêng của nó và chỉ có nó có quyền truy cập vào vùng dữ liệu này. Do các microservice là độc lập, chúng không giao tiếp trực tiếp với nhau mà qua một thành phần trung gian được gọi là API gateway. Có thể thấy vai trò của API gateway rất quan trọng trong mô hình microservice. Nó là điểm đến và đi của mọi yêu cầu hay phản hồi.
\par
Tính phân tán là đặc trưng của kiến trúc này, vì vậy việc xác định mức độ chi
tiết của từng dịch vụ là chìa khóa để xây dựng một hệ thống tốt. Đò là điều khó
đạt được đối với hệ thống đang phát triển của chúng tôi. Một vấn đề khác với
hướng tiếp cận theo kiến trúc này đó là việc các nghiệp vụ trong hệ thống chúng
tôi liên quan khá chặt chẽ, vì vậy kiến trúc Microservices tỏ ra không phù hợp
với hệ thống của chúng tôi.