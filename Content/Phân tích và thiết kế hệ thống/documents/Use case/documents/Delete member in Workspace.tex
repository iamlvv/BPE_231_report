\subsubsection{Xóa thành viên khỏi workspace}

\begin{table}[H]
\def\arraystretch{2}%
\centering
\resizebox{\textwidth}{!}{%
\begin{tabular}{|p{3cm}|p{11cm}|}
\hline
\textbf{Use-case name} &
  Xóa thành viên khỏi workspace
\\ \hline
\textbf{Actor} &
  Người sở hữu workspace
\\ \hline
\textbf{Description} &
  Người sở hữu workspace xóa thành viên ra khỏi Workspace 
\\ \hline
\textbf{Trigger} &
  Không 
\\ \hline
\textbf{Pre-conditions} &
  Người dùng đăng nhập vào hệ thống, người dùng hiện tại đang ở giao diện workspace mà người dùng sở hữu
\\ \hline
\textbf{Post-conditions} &
  Thành công xóa người dùng khỏi workspace
\\ \hline
\textbf{Normal Flow} &
  \begin{enumerate}
      \item Người dùng chọn icon hình bánh răng bên cạnh tiêu đề workspace, hệ thống chuyển hướng đến trang quản lý workspace. Mặc định ở giao diện "Members management"
      \item Hệ thống hiển thị danh sách thành viên trong workspace.
      \item Người dùng chọn một hoặc nhiều thành viên. Hệ thống hiển thị nút "Delete"
      \item Người dùng chọn nút “Delete”, hệ thống hiển thị modal xác nhận thao tác xóa thành viên đã chọn khỏi workspace
      \item Người dùng chọn “Delete” để loại trừ thành viên khỏi workspace
  \end{enumerate}
  \\ \hline
\textbf{Alternative Flow} &
    \begin{tabular}{p{10cm}}
        3a. Danh sách thành viên chỉ có workspace owner, người dùng không thể lựa chọn thành viên để thao tác
    \end{tabular}
\\ \hline
\textbf{Exceptions} &
     \begin{tabular}{p{10cm}}
        5a. Người dùng chọn "Cancel" để tắt modal
     \end{tabular}
\\ \hline
\end{tabular}%
}
\caption{Use-case scenario cho use-case Xóa thành viên khỏi workspace}
\end{table}