\subsection{Di chuyển câu hỏi trong bảng khảo sát}
\begin{table}[H]
\def\arraystretch{0.5}%
\centering
\resizebox{\textwidth}{!}{%
\fontsize{11}{13}\selectfont
\begin{tabular}{|p{3cm}|p{11cm}|}
\hline
\textbf{Use-case name}     & Di chuyển câu hỏi trong bảng khảo sát                          \\ \hline
\textbf{Actor}             & Người sở hữu Project    \\ \hline
\textbf{Description}       & Người sở hữu Project di chuyển câu hỏi trong phần của bảng khảo sát                      \\ \hline
\textbf{Trigger}           & Không                                    \\ \hline
\textbf{Pre-conditions} &
  Người dùng đã đăng ký vào hệ thống và đăng nhập thành công, hiện tại đang đứng ở giao diện \textbf{Survey Builder} \\ \hline
\textbf{Post-conditions}   & Người dùng di chuyển câu hỏi thành công \\ \hline
\textbf{Normal flow} &
\begin{tabular}{p{10.5cm}}
  \begin{enumerate}
      \item Người dùng nhấn giữ 1 câu hỏi trong bảng khảo sát.
      \item Người dùng kéo thả câu hỏi tới vị trí mới.
      \item Hệ thống xác nhận thao tác và đặt câu hỏi ở vị trí mới.
  \end{enumerate}
\end{tabular}
  \\ \hline
\textbf{Alternative flows} & 
    \begin{tabular}{p{10cm}}
        1a. Người dùng chọn 1 câu hỏi trong bảng khảo sát. \\
        2a. Hệ thống hiển thị chi tiết câu hỏi ở sidebar bên cạnh. \\
        3a. Người dùng chọn biểu tượng dấu cộng hoặc trừ ở mục \textbf{Position in section} để di chuyển câu hỏi lên xuống. \\
        4a. Người dùng chọn Apply để xác nhận thay đổi. \\
        Tiếp tục Normal Flow ở bước 3.
    \end{tabular}                 \\ \hline
\textbf{Exceptions}        & Không                                    \\ \hline
\end{tabular}%
}
\caption{Use-case scenario cho use-case Di chuyển câu hỏi trong bảng khảo sát}
\end{table}