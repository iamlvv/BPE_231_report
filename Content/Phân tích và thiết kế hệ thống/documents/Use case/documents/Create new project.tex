\subsection{Tạo mới project trong workspace}

\begin{table}[H]
\def\arraystretch{2}%
\centering
\resizebox{\textwidth}{!}{%
\begin{tabular}{|p{3cm}|p{11cm}|}
\hline
\textbf{Use-case name}    & Tạo mới project trong workspace
\\ \hline
\textbf{Actor}            & Người tham gia vào workspace có quyền hạn chỉnh sửa
\\ \hline
\textbf{Description}      & Người dùng có thể tạo project trong Workspace 
\\ \hline
\textbf{Trigger}          & Không
\\ \hline
\textbf{Pre-conditions}   & Người dùng đã đăng nhập vào hệ thống, hiện tại đang ở giao diện workspace người dùng muốn tạo project. Người dùng có quyền hạn của editor hoặc owner.
\\ \hline
\textbf{Post-conditions}  & Tạo project thành công
\\ \hline
\textbf{Normal Flow} &
    \begin{enumerate}
        \item Người dùng chọn nút “Create project”
        \item Hệ thống mở modal yêu cầu thông tin của project: project name, project description
        \item Người dùng nhập tên (Name) của project
        \item Người dùng nhập mô tả (Description) của project
        \item Người dùng chọn nút “Save” để lưu thông tin và hoàn tất tạo mới project
    \end{enumerate}
\\ \hline
\textbf{Alternative Flow} & Không
\\ \hline
\textbf{Exceptions} &
   \begin{tabular}{p{10cm}}
      3a. Người dùng chọn nút “Cancel” để đóng modal nhập thông tin workspace
      \\
      4a. Người dùng sử dụng tên của workspace đã tồn tại
      \\ 
      \begin{tabular}{p{9cm}}
          4a.1. Hệ thống yêu cầu người dùng sửa lại tên phù hợp
          \\ 
          4a.2. Tiếp tục bước 4
      \end{tabular}
      \\
      4b. Người dùng không nhập tên của workspace
      \\ 
      \begin{tabular}{p{9cm}}
          4b.1. Hệ thống yêu cầu người dùng phải điền thông tin bắt buộc
          \\ 
          4b.2. Tiếp tục bước 4
      \end{tabular}
  \end{tabular} 
\\ \hline
\end{tabular}%
}
\caption{Use-case scenario cho use-case Tạo mới project trong workspace}
\end{table}