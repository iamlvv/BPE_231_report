\subsection{Xóa project trong Workspace}

\begin{table}[H]
\def\arraystretch{2}%
\centering
\resizebox{\textwidth}{!}{%
\begin{tabular}{|p{3cm}|p{11cm}|}
\hline
\textbf{Use-case name} & Xóa project trong workspace 
\\ \hline
\textbf{Actor} & Người tham gia vào workspace có quyền hạn chỉnh sửa
\\ \hline
\textbf{Description} & Người dùng có thể xóa project trong Workspace
\\ \hline
\textbf{Trigger} & Không 
\\ \hline
\textbf{Pre-conditions} & Người dùng đã đăng nhập vào hệ thống và đang ở giao diện workspace đã chọn, người dùng có quyền hạn của editor hoặc owner
\\ \hline
\textbf{Post-conditions} & Xóa project thành công 
\\ \hline
\textbf{Normal Flow} &
    \begin{enumerate}
        \item Người dùng hiện tại đang ở giao diện workspace và hệ thống hiển thị danh sách project.
        \item Người dùng chọn vào icon "Menu" của item project, hệ thống hiển thị danh sách menu thao tác với project
        \item Người dùng chọn nút "Delete" trong menu
        \item Hệ thống hiển thị modal yêu cầu người dùng xác nhận xóa project
        \item Người dùng nhấn nút “Delete” để xác nhận xóa project
    \end{enumerate}
\\ \hline
\textbf{Alternative Flow} & Không 
\\ \hline
\textbf{Exceptions} &
  \begin{tabular}{p{10cm}}
        5a. Người dùng chọn "Cancel" để hủy thao tác xóa project
  \end{tabular}
\\ \hline
\end{tabular}%
}
\caption{Use-case scenario cho use-case Xóa project trong Workspace}
\end{table}