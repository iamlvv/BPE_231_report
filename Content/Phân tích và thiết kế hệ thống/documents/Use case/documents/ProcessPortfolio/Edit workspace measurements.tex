\subsection{Chỉnh sửa giá trị thang đo của workspace}

\begin{table}[H]
\def\arraystretch{2}%
\centering
\resizebox{\textwidth}{!}{%
\begin{tabular}{|p{3cm}|p{11cm}|}
  \hline
  \textbf{Use-case name}     & Chỉnh sửa giá trị thang đo của workspace
  \\ \hline
  \textbf{Actor}             & Chủ sở hữu workspace
  \\ \hline
  \textbf{Description}       & Thay đổi giá trị thang đo của workspace
  \\ \hline
  \textbf{Trigger}           & Không                                                      
  \\ \hline
  \textbf{Pre-conditions}    & Người dùng đã đăng nhập vào hệ thống. Người dùng sở hữu một workspace và đang truy cập vào giao diện quản lý workspace (workspace management)   
  \\ \hline
  \textbf{Post-conditions}   & Người dùng thành công thay đổi giá trị thang đo của workspace, thang đo này áp dụng cho toàn bộ quy trình trong workspace. 
  \\ \hline
  \textbf{Normal flow} &
    \begin{enumerate}
        \item Người dùng chọn “Process portfolio” trên sidebar, hệ thống điều hướng tới giao diện quản lý process portfolio.
        \item Người dùng chọn nút “Performance level”, hệ thống mở modal “Configure performance level”.
        \item Người dùng nhập thông tin được yêu cầu.
        \item Người dùng chọn nút “Save” để lưu thông tin về thang đo, hệ thống đóng modal.
    \end{enumerate}
  \\ \hline
  \textbf{Alternative flows} & 
    \begin{tabular}{p{10cm}}
      3a. Người dùng không nhập thông tin, chọn nút “Save” để lưu thông tin về thang đo, hệ thống đóng modal.
    \end{tabular}
  \\ \hline
  \textbf{Exceptions}        &
    \begin{tabular}{p{10cm}}
      3. Người dùng nhập kiểu dữ liệu không hợp lệ
      \begin{enumerate}[label=\alph*)]
        \item Người dùng chọn nút “Save” để lưu lại thông tin về thang đo, hệ thống hiện cảnh báo
        \item Quay lại bước 2
      \end{enumerate}
      \\
      3. Người dùng nhập giá trị Target cycle time/cost lớn hơn giá trị Worst cycle time/cost hoặc nhập giá trị Target quality/flexibility nhỏ hơn Worst quality/flexibility
      \begin{enumerate}[label=\alph*)]
        \item Người dùng chọn nút “Save” để lưu lại thông tin về thang đo, hệ thống hiện cảnh báo
        \item Quay lại bước 2
      \end{enumerate}
    \end{tabular}
  \\ \hline
\end{tabular}%
}
\caption{Use-case scenario cho use-case Chỉnh sửa giá trị thang đo của workspace}
\end{table}

\begin{table}[H]
  \def\arraystretch{2}%
  \centering
  \resizebox{\textwidth}{!}{%
    \begin{tabular}{|p{3cm}|p{11cm}|}
      \hline
      \textbf{Tên trường} & \textbf{Mô tả}                                                  \\ \hline
      Target cycle time   & Giá trị độ đo thời gian người dùng mong muốn đạt được           \\ \hline
      Worst cycle time    & Giá trị độ đo thời gian người dùng tối đa cho phép đạt được     \\ \hline
      Target cost         & Giá trị độ đo chi phí người dùng mong muốn đạt được             \\ \hline
      Worst cost          & Giá trị độ đo chi phí người dùng tối đa cho phép đạt được       \\ \hline
      Target quality      & Giá trị độ đo chất lượng người dùng mong muốn đạt được          \\ \hline
      Worst quality       & Giá trị độ đo chất lượng người dùng tối thiểu cho phép đạt được \\ \hline
      Target flexibility  & Giá trị độ đo linh hoạt người dùng mong muốn đạt được           \\ \hline
      Worst flexibility   & Giá trị độ đo linh hoạt người dùng tối thiểu cho phép đạt được  \\ \hline
    \end{tabular}%
  }
  \caption{Thông tin về yêu cầu đầu vào của use-case Chỉnh sửa giá trị thang đo của workspace}
\end{table}