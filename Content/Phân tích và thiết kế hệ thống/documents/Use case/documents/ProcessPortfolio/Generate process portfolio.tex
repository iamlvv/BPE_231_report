\subsection{Khởi tạo process portfolio}

\begin{table}[H]
\def\arraystretch{2}%
\centering
\resizebox{\textwidth}{!}{%
\begin{tabular}{|p{3cm}|p{11cm}|}
  \hline
  \textbf{Use-case name}     & Khởi tạo process portfolio
  \\ \hline
  \textbf{Actor}             & Chủ sở hữu workspace
  \\ \hline
  \textbf{Description}       & Khởi tạo process portfolio cho workspace
  \\ \hline
  \textbf{Trigger}           & Không                                                      
  \\ \hline
  \textbf{Pre-conditions}    & Người dùng đã đăng nhập vào hệ thống. Người dùng sở hữu một workspace và đang truy cập vào giao diện quản lý workspace (workspace management)
  \\ \hline
  \textbf{Post-conditions}   & Người dùng thành công khởi tạo biểu đồ process portfolio của workspace
  \\ \hline
  \textbf{Normal flow} &
    \begin{enumerate}
      \item Người dùng chọn “Process portfolio” trên sidebar, hệ thống điều hướng tới giao diện quản lý process portfolio.
      \item Người dùng chọn nút “Process portfolio”.
      \item Hệ thống mở modal và khởi tạo process portfolio.
      \item Người dùng chọn “Close” để đóng modal      
    \end{enumerate}
  \\ \hline
  \textbf{Alternative flows} & 
    \begin{tabular}{p{10cm}}
      3. Trong workspace còn có những active version còn bị khuyết giá trị
      \begin{enumerate}[label=\alph*)]
        \item Hệ thống hiển thị danh sách những active version còn bị khuyết giá trị (feasibility/health/strategic importance)
        \item Người dùng chọn icon “ba dấu chấm” và chọn “Edit”, hệ thống mở modal cho phép người dùng \item chỉnh sửa giá trị health/feasibility/strategic importance của process version tương ứng.
        \item Người dùng nhập thông tin được yêu cầu.
        \item Người dùng chọn nút “Save” để lưu thông tin của process version, hệ thống đóng modal Edit.
        \item Quay lại bước 3
      \end{enumerate}
    \end{tabular}
  \\ \hline
  \textbf{Exceptions}        & Không
  \\ \hline
\end{tabular}%
}
\caption{Use-case scenario cho use-case Khởi tạo process portfolio}
\end{table}