\subsection{Chỉnh sửa thông tin của process version}

\begin{table}[H]
\def\arraystretch{0.5}%
\centering
\resizebox{\textwidth}{!}{%
\fontsize{11}{13}\selectfont
\begin{tabular}{|p{3cm}|p{11cm}|}
  \hline
  \textbf{Use-case name}     & Chỉnh sửa thông tin của process version
  \\ \hline
  \textbf{Actor}             & Chủ sở hữu workspace
  \\ \hline
  \textbf{Description}       & Thay đổi thông tin của process version như các giá trị health/feasibility/strategic importance của process version
  \\ \hline
  \textbf{Trigger}           & Không                                                      
  \\ \hline
  \textbf{Pre-conditions}    & 
  \begin{tabular}{p{10.5cm}}
    Người dùng đã đăng nhập vào hệ thống. Người dùng sở hữu một workspace và đang truy cập vào giao diện quản lý workspace (workspace management)
  \end{tabular}
  \\ \hline
  \textbf{Post-conditions}   &
  \begin{tabular}{p{10.5cm}}
    Người dùng thành công thay đổi giá trị thay đổi giá trị health/feasibility/strategic importance của process version, những giá trị này sẽ được sử dụng để khởi tạo process portfolio.
  \end{tabular}
  \\ \hline
  \textbf{Normal flow} &
  \begin{tabular}{p{10.5cm}}
    \begin{enumerate}
        \item Người dùng chọn “Process portfolio” trên sidebar, hệ thống điều hướng tới giao diện quản lý process portfolio.
        \item Hệ thống hiển thị danh sách project/process/process version có trong workspace.
        \item Người dùng chọn icon “ba dấu chấm” và chọn “Edit”, hệ thống mở modal cho phép người dùng chỉnh sửa giá trị health/feasibility/strategic importance của process version tương ứng.
        \item Người dùng nhập thông tin được yêu cầu.
        \item Người dùng chọn nút “Save” để lưu thông tin của process version, hệ thống đóng modal.
    \end{enumerate}
  \end{tabular}
  \\ \hline
  \textbf{Alternative flows} & 
    \begin{tabular}{p{10.5cm}}
      4. Người dùng không nhập thông tin. Người dùng chọn nút “Save” để lưu thông tin về thang đo, hệ thống đóng modal.
    \end{tabular}
  \\ \hline
  \textbf{Exceptions}        &
    \begin{tabular}{p{10.5cm}}
      3. Người dùng nhập kiểu dữ liệu không hợp lệ
      \begin{enumerate}[label=\alph*)]
        \item Người dùng chọn nút “Save” để lưu lại thông tin về thang đo, hệ thống hiện cảnh báo
        \item Quay lại bước 3
      \end{enumerate}
    \end{tabular}
  \\ \hline
\end{tabular}%
}
\caption{Use-case scenario cho use-case Chỉnh sửa thông tin của process version}
\end{table}

\begin{table}[H]
  \def\arraystretch{0.5}%
  \centering
  \resizebox{\textwidth}{!}{%
  \fontsize{11}{13}\selectfont
    \begin{tabular}{|p{4cm}|p{10cm}|}
      \hline
      \textbf{Tên trường} & \textbf{Mô tả}                                                           \\ \hline
      Current cycle time  & Giá trị độ đo thời gian của quy trình hiện tại được người dùng cung cấp  \\ \hline
      Current cost        & Giá trị độ đo chi phí của quy trình hiện tại được người dùng cung cấp    \\ \hline
      Current quality     & Giá trị độ đo chất lượng của quy trình hiện tại được người dùng cung cấp \\ \hline
      Current flexibility & Giá trị độ đo linh hoạt của quy trình hiện tại được người dùng cung cấp  \\ \hline
      Feasibility         & Đánh giá tính khả thi của quy trình hiện tại được người dùng cung cấp    \\ \hline
      Strategic importance & Đánh giá mức độ quan trọng chiến lược của quy trình hiện tại được người dùng cung cấp \\ \hline
      Target flexibility  & Giá trị độ đo linh hoạt người dùng mong muốn đạt được                    \\ \hline
      Worst flexibility   & Giá trị độ đo linh hoạt người dùng tối thiểu cho phép đạt được           \\ \hline
    \end{tabular}%
  }
  \caption{Thông tin về yêu cầu đầu vào của use-case Chỉnh sửa thông tin của process version}
\end{table}

\begin{table}[H]
  \def\arraystretch{0.5}%
  \centering
  \resizebox{\textwidth}{!}{%
  \fontsize{11}{13}\selectfont
    \begin{tabular}{|p{4cm}|p{10cm}|}
      \hline
      \textbf{Tên trường} & \textbf{Mô tả}                                                           \\ \hline
      Evaluated cycle time  & Giá trị độ đo thời gian của quy trình hiện tại được hệ thống tính toán dựa trên mô hình của quy trình  \\ \hline
      Evaluated cost        & Giá trị độ đo chi phí của quy trình hiện tại được hệ thống tính toán dựa trên mô hình của quy trình    \\ \hline
      Evaluated quality     & Giá trị độ đo chất lượng của quy trình hiện tại được hệ thống tính toán dựa trên mô hình của quy trình \\ \hline
      Evaluated flexibility & Giá trị độ đo linh hoạt của quy trình hiện tại được hệ thống tính toán dựa trên mô hình của quy trình  \\ \hline
    \end{tabular}%
  }
  \caption{Thông tin hệ thống cung cấp của use-case Chỉnh sửa thông tin của process version}
\end{table}