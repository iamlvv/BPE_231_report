\subsection{Gửi yêu cầu điều chỉnh quyền hạn trong workspace}

\begin{table}[H]
\def\arraystretch{0.5}%
\centering
\resizebox{\textwidth}{!}{%
\fontsize{11}{13}\selectfont
\begin{tabular}{|p{3cm}|p{11cm}|}
\hline
\textbf{Use-case name} &
  Gửi yêu cầu điều chỉnh quyền hạn trong workspace
\\ \hline
\textbf{Actor} &
  Thành viên đã tham gia vào workspace
\\ \hline
\textbf{Description} &
  Gửi yêu cầu “Cung cấp thêm quyền” tới workspace owner để được xét duyệt 
\\ \hline
\textbf{Trigger} &
  Không
\\ \hline
\textbf{Pre-conditions} &
  Người dùng đã đăng ký vào hệ thống và đăng nhập thành công
\\ \hline
\textbf{Post-conditions} &
  Người dùng gửi thành công yêu cầu “Cung cấp điều chỉnh quyền” tới workspace owner để được xét duyệt 
\\ \hline
\textbf{Normal flow} &
\begin{tabular}{p{10.5cm}}
  \begin{enumerate}
      \item Người dùng ở giao diện của workspace đã chọn và đang tham gia vào workspace dưới quyền hạn của viewer hoặc sharer
      \item Người dùng chọn nút “Create new project”, hệ thống hiện modal thông báo:
      \begin{enumerate}
          \item Người dùng không có quyền hạn để chỉnh sửa nội dung workspace
          \item Gửi yêu cầu đến workspace owner để điều chỉnh quyền trong workspace
      \end{enumerate}
      \item Người dùng chọn nút “Send” để gửi yêu cầu
  \end{enumerate}
\end{tabular}
\\ \hline
\textbf{Alternative flows} &
  Không 
\\ \hline
\textbf{Exceptions} &
  \begin{tabular}{p{10.5cm}}
      3a. Người dùng chọn “Cancel” để đóng modal
  \end{tabular} 
\\ \hline
\end{tabular}%
}
\caption{Use-case scenario cho use-case Gửi yêu cầu điều chỉnh quyền hạn trong workspace}
\end{table}