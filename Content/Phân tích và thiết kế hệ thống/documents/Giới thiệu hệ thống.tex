\section{Giới thiệu hệ thống}
Hệ thống BPSky trong giai đoạn trước đã thành công hiện thực bộ công cụ cho phép người dùng thiết kế quy trình dưới dạng mô hình thông qua ngôn ngữ mô hình hóa BPMN và đánh giá chất lượng quy trình một cách tự động. Bên cạnh đó, bộ công cụ trong hệ thống BPSky cũng cung cấp những chức năng khác như hỗ trợ nhập xuất file định dạng .bpmn, ghi chú trên các thành phần và mô hình, các tính năng undo, redo, zoom-in, zoom-out,... Trong quá trình thiết kế, người dùng cũng có thể xác thực mô hình BPMN về mặt ngữ nghĩa cũng như so sánh các phiên bản khác nhau của mô hình.
\par
Trong giai đoạn này, hệ thống sẽ được tập trung cải tiến về phạm vi quản lý và lưu trữ của quy trình nghiệp vụ. Các dự án của người dùng sẽ được mở rộng từ không gian cá nhân thành không gian làm việc chung, cho phép người dùng chia sẻ đến với các thành viên khác trong tổ chức. Khi mở rộng không gian làm việc thì hệ thống cũng sẽ đáp ứng những nhu cầu về quản lý thành viên, quản lý yêu cầu từ thành viên và vấn đề giám sát hiệu suất của quy trình.

Ngoài ra hệ thống cũng mong muốn bổ sung vào tính năng đánh giá chất lượng quy trình bằng việc tích hợp tính năng tạo khảo sát vào bộ công cụ của hệ thống, không chỉ đánh giá chất lượng thông qua mô hình của quy trình mà còn bổ sung những đánh giá từ phía người tham gia vận hành quy trình và sử dụng sản phẩm của quy trình.
\par
Những tính năng trên sẽ được cụ thể hóa thông qua các nhóm chức năng sau:
\begin{enumerate}
      \item Quản lý không gian làm việc chung: Cho phép người dùng tạo không gian làm việc và chia sẻ đến với các người dùng khác trong hệ thống.
      \item Quản lý thành viên trong không gian làm việc: Quản lý phân quyền và thông tin của những thành viên được mời vào workspace.
      \item Quản lý yêu cầu được gửi từ thành viên: Các yêu cầu liên quan tới việc phân quyền và mời người dùng khác vào không gian làm việc sẽ được người sở hữu workspace xem xét và phản hồi.
      \item Quản lý danh mục quy trình: Cho phép người dùng chỉnh sửa các thông số của quy trình liên quan tới tính khả thi, mức độ ảnh hưởng tới chiến lược tổ chức và hiệu suất của quy trình; từ đó tạo danh mục quy trình cho không gian làm việc để hỗ trợ khả năng giám sát.
      \item Quản lý thông báo cá nhân: Cho phép người dùng nhận lời mời thông qua hòm thông báo của người dùng. Ngoài ra cũng có thể nhận các thông báo từ hệ thống theo thời gian thực.
      \item Tích hợp khảo sát vào đánh giá chất lượng quy trình: Bên cạnh chất lượng được hệ thống tính toán dựa trên mô hình quy trình nghiệp vụ, người dùng có thể tạo khảo sát cho từng quy trình để thu thập ý kiến của người dùng khác.
\end{enumerate}
\par
Các yêu cầu phi chức năng cũng là yếu tố quan trọng cần được đánh giá, xem xét. Các yêu cầu phi chức năng dưới đây được chúng tôi đảm bảo hệ thống có thể thực hiện được nhằm tối ưu hoá trải nghiệm của người dùng, bên cạnh các yêu cầu phi chức năng kế thừa từ đề tài trước đó:

\begin{table}[H]
      \def\arraystretch{2}%
      \centering
      \resizebox{\textwidth}{!}{
      \begin{tabular}{ |m{4cm}|m{10cm}|}
            \hline
            {\centering Hiệu suất} & 
            \begin{itemize}
                  \item Đối với màn hình input: tối đa 5 trường dữ liệu, không tính toán dữ liệu phức tạp, không tương tác với hệ thống ngoài, có thể lưu trữ dữ liệu trực
                        tiếp ngay xuống database, không lưu trữ các tệp nội dung lớn như hình ảnh, video, tập tin quá 3 MB.
                  \item Đối với màn hình output: dữ liệu được truy vấn trực tiếp từ database, hạn chế những câu lệnh truy vấn phức tạp, những truy vấn từ hệ thống ngoài. Hiển thị tối đa 10 dòng dữ liệu, có độ dài nhỏ hơn 100 ký tự
            \end{itemize} 
            \\ \hline
            {\centering Tính khả dụng} & 
            \begin{itemize}
                  \item Tất cả thông tin quan trọng phải được hiển thị trong 1 màn hình (không cần phải
                        thực hiện thêm thao tác cuộn).
                  \item Giao diện của hệ thống cần có sự nhất quán, về mặt hình ảnh biểu tượng cũng như
                        vị trí các đối tượng trên màn hình để người dùng làm quen dễ dàng hơn.
                  \item Người dùng có thể thành thạo các thao tác trên màn hình trong 15 phút sử dụng.
            \end{itemize}
            \\ \hline
            {\centering Khả năng hỗ trợ trình duyệt} & 
            \begin{itemize}
                  \item Hệ thống có thể được sử dụng hiệu quả trên các trình duyệt web (Opera, UC
                        Browser, Safari, Microsoft Edge, Google Chrome, Samsung Internet Browser).
                  \item Hệ thống không chạy trên các phiên bản trình duyệt quá cũ.
            \end{itemize} 
            \\ \hline
            {\centering Khả năng mở rộng} & 
            \begin{itemize}
                  \item Triển khai cơ sở dữ liệu trên một máy chủ riêng và phần backend của hệ thống trên một máy chủ riêng
            \end{itemize}
            \\ \hline
      \end{tabular}
      }
      \caption{Bảng liệt kê tiêu chí về yêu cầu phi chức năng của hệ thống}
\end{table}
