\subsection{Giới thiệu hệ thống}
Ở đề tài trước, hệ thống (tạm gọi là phiên bản 1.0) đã được hiện thực với một số tính năng quan trọng, chẳng hạn như: thiết kế, so sánh, mô phỏng và đánh giá mô hình BPMN. Ngoài việc cung cấp bộ công cụ đầy đủ các thành phần BPMN 2.0, phiên bản 1.0 này của hệ thống còn hỗ trợ nhập xuất file BPMN, ghi chú trên các thành phần và mô hình, các tính năng undo, redo, zoom-in, zoom-out,... Trong quá trình thiết kế, người dùng cũng có thể xác thực mô hình BPMN về mặt ngữ nghĩa cũng như so sánh các phiên bản khác nhau của mô hình.
\newline
Nhóm chúng tôi nhận ra rằng tính năng quản lý mô hình BPMN của người dùng vẫn chưa được chú trọng. Các dự án của người dùng chỉ mới được đóng gói trong các đơn vị gọi là Project, và chỉ mới thuộc về riêng một cá nhân sở hữu project đó. Tuy nhiên, khi hệ thống được mở rộng, đồng nghĩa với việc số lượng người dùng tăng lên, người ta sẽ có xu hướng nhóm các dự án lại với nhau thành từng cụm lớn hơn. Trong những cụm đó, những người dùng có thể làm việc với nhau trên các project bên trong mỗi cụm, giúp phân tách và quản lý dự án hiệu quả hơn. Chúng tôi gọi đó là tính năng Workspace.
\newline
Workspace được cụ thể hoá bởi các nhóm tính năng chính sau đây:
\begin{itemize}
    \item Quản lý thành viên (members management). Một Workspace có thể có một hoặc nhiều
          người cùng tham gia. Mỗi người dùng đều có một Workspace cá nhân của bản thân,
          gọi là Personal Workspace. Personal Workspace là riêng tư, nên không thể mời
          người khác tham gia vào Personal Workspace của bản thân. Người dùng có thể tự
          tạo Workspace khác (tạm gọi là Regular Workspace) để có thể làm việc với nhiều
          người cho các dự án ở trong Workspace đó. Chủ sở hữu Workspace có thể mời người
          khác tham gia vào Workspace của mình, cũng như thiết lập một số quyền hạn truy
          cập của thành viên, hay loại bỏ thành viên ra khỏi Workspace.
    \item Quản lý dự án (projects management). Các Projects sẽ được khởi tạo trong
          Workspace, cho phép các thành viên trong Workspace có thể truy cập các project
          liên quan tới Workspace đó. Bất kỳ thành viên nào của Workspace với bất kỳ
          quyền hạn nào cũng có thể tạo project của riêng mình trong các Workspace mà
          mình tham gia, và mời các thành viên khác vào project được tạo để cùng nhau
          phát triển. Các projects cũng có thể bị xoá khỏi Workspace tuỳ theo mục đích và
          nhu cầu.
    \item Quản lý yêu cầu (requests management). Quản lý requests là nhóm các tính năng
          cho phép chủ sở hữu Workspace có thể quản lý được những yêu cầu tham gia vào
          Workspace hay những yêu cầu thay đổi quyền hạn của bản thân thành viên để có
          thêm một số quyền truy cập đối với các dữ liệu trong Workspace. Chủ sở hữu
          Workspace có thể chấp thuận, từ chối hay xoá bỏ những yêu cầu nào từ phía thành
          viên. Khi một request được chấp thuận hay từ chối, sẽ có một thông báo gửi về
          cho đối tượng của request đó. Bên cạnh đó, chủ sở hữu Workspace cũng sẽ nhận
          được các requests theo thời gian thực, ngay khi thành viên gửi đi bất kỳ yêu
          cầu nào.
    \item Cá nhân hoá Workspace. Hệ thống cho phép chủ sở hữu Workspace linh hoạt tuỳ
          chỉnh một số thuộc tính của Workspace như tên, mô tả, hình ảnh đại diện của
          Workspace nhằm tăng trải nghiệm cá nhân và phù hợp với mục đích của Workspace
          đó.
\end{itemize}
Bên cạnh tính năng Workspace, nhóm chúng tôi còn phát triển thêm những tính năng thông báo:
\begin{itemize}
    \item Quản lý các thông báo cá nhân. Mỗi người dùng sẽ có một bảng thông báo tổng hợp
          của riêng mình. Các loại thông báo hiện tại bao gồm: thông báo khi yêu cầu thay đổi quyền hạn của bản thân trong
          Workspace tham gia được chấp thuận hay từ chối, thông báo lời mời tham gia
          Workspace từ một thành viên trong Workspace đó. Người dùng có thể chấp nhận
          hoặc từ chối lời mời này. Bên cạnh đó, người dùng có thể thực thi các thao tác
          xoá, lọc, tìm kiếm, đánh dấu những thông báo quan trọng.
    \item Nhận, gửi thông báo theo thời gian thực. Thông thường, để luôn nhận được thông
          báo mới nhất, người dùng cần phải tải lại trang định kỳ để hệ thống trả về các
          thông báo mới. Tuy nhiên, chúng tôi đã hiện thực tính năng thông báo này
          theo thời gian thực, nghĩa là khi có một thông báo mới, nó sẽ được hiển thị
          đồng thời, ngay lập tức cho người dùng, để họ dù đang làm gì trên hệ thống vẫn
          có thể theo dõi được các yêu cầu thay đổi quyền hạn của mình có được chấp nhận hay chưa, hay nhận
          lời mời tham gia Workspace ngay lập tức. Việc hiện thực theo thời gian thực
          giúp tăng khả năng phản hồi và trải nghiệm của người dùng khi sử dụng hệ thống.
\end{itemize}
\par
Trong giai đoạn tới, dựa trên nền tảng lý thuyết đã được xây dựng ở trên, chúng tôi
sẽ hiện thực bảng khảo sát đánh giá sự hài lòng của khách hàng và người dùng tham gia
thực thi quy trình đối với chất lượng của quy trình nghiệp vụ, từ đó đánh giá lại chất
lượng tổng thể của quy trình. Đồng thời, chúng tôi cũng sẽ phát triển thêm tính năng
giám sát, cải tiến quy trình nghiệp vụ, trực quan hoá mức độ ưu tiên của các quy trình
bên trong tổ chức, hỗ trợ đưa ra quyết định cho sự phát triển của quy trình trong tương lai.
\par
Không những vậy, chúng tôi cũng sẽ hiện thực tính năng Template Suggestions. 
Chúng tôi đang hướng tới việc mở hoá hệ thống, nghĩa là cho phép người dùng có thể 
công khai một số project đến cho cộng đồng để được đóng góp và phát triển thành 
những sản phẩm tiềm năng, tối ưu. Những người dùng khác có thể tham gia phát triển, 
tạo bản sao đem về project cá nhân của họ và đánh giá những dự án chất lượng cao 
nhằm xây dựng tài nguyên các mô hình BPMN và đưa ra các đề xuất phù hợp với nhu 
cầu của người dùng khi họ cần tìm kiếm hay khởi tạo dự án nào trong hệ thống, họ 
có thể tham khảo tới những mẫu mô hình được đánh giá cao đã có sẵn.
\newline

Các yêu cầu phi chức năng cũng là yếu tố quan trọng cần được đánh giá, xem xét.
Các yêu cầu phi chức năng dưới đây được chúng tôi đảm bảo hệ thống có thể thực
hiện được nhằm tối ưu hoá trải nghiệm của người dùng, bên cạnh các yêu cầu phi
chức năng kế thừa từ đề tài trước đó:
\begin{center}
    \begin{tabular}{ |m{2.5cm}|m{11.5cm}|}
        \hline
        Performance    & 
      \begin{itemize}
            \item Đối với màn hình input: tối đa 5 trường dữ liệu, không tính toán dữ liệu phức tạp, không tương tác với hệ thống ngoài, có thể lưu trữ dữ liệu trực
                  tiếp ngay xuống database, không lưu trữ các tệp nội dung lớn như hình ảnh, video, tập tin quá 3 MB.
            \item Đối với màn hình output: dữ liệu được truy vấn trực tiếp từ database, hạn chế những câu lệnh truy vấn phức tạp, những truy vấn từ hệ thống ngoài.
                  Hiển thị tối đa 10 dòng dữ liệu, có độ dài nhỏ hơn 100 ký tự
      \end{itemize} \\
        \hline
        Usability      & 
        \begin{itemize}
            \item Tất cả thông tin quan trọng phải được hiển thị trong 1 màn hình (không cần phải
                  thực hiện thêm thao tác cuộn).
            \item Giao diện của hệ thống cần có sự nhất quán, về mặt hình ảnh biểu tượng cũng như
                  vị trí các đối tượng trên màn hình để người dùng làm quen dễ dàng hơn.
            \item Người dùng có thể thành thạo các thao tác trên màn hình trong 15 phút sử dụng.
      \end{itemize}\\
        \hline
        Supportability & 
            \begin{itemize}
                  \item Hệ thống có thể được sử dụng hiệu quả trên các trình duyệt web (Opera, UC
                        Browser, Safari, Microsoft Edge, Google Chrome, Samsung Internet Browser).
                  \item Hệ thống không chạy trên các phiên bản trình duyệt quá cũ.
            \end{itemize}\\
        \hline
        Scalability    & 
            \begin{itemize}
                  \item Có khả năng tách database trên một server riêng và backend trên một
                  server riêng
            \end{itemize}\\
        \hline
    \end{tabular}
    \caption{Bảng liệt kê tiêu chí về yêu cầu phi chức năng của hệ thống}
\end{center}
