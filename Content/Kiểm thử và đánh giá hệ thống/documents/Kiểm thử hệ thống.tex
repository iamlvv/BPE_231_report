\subsection{Tổng quan các phương pháp kiểm thử}

\subsubsection{Usecase testing}
Kiểm thử phần mềm tập trung vào việc kiểm tra các tình huống hoặc trường hợp sử dụng cụ thể của một phần mềm, được quy định trong usecase scenario.

\begin{itemize}
    \item Ưu điểm
        \subitem{Usecase testing thường dựa trên các kịch bản sử dụng đơn giản và dễ hiểu, giúp cho quá trình kiểm thử trở nên dễ dàng hơn cho cả nhóm phát triển và nhóm kiểm thử.}
        \subitem{Unit testing là bước đệm giúp chúng ta thực hiện integration test. Phát hiện lỗi sớm sẽ giúp tiết kiệm thời gian và chi phí của cả quá trình kiểm thử về sau.}
        \subitem{Hỗ trợ việc tái sử dụng testcase.}
    \item Nhược điểm
        \subitem{Nếu usecase scenario không được viết rõ ràng hoặc không đầy đủ, việc thực hiện usecase testing có thể trở nên khó khăn và không chính xác.}
        \subitem{Usecase cũng tập trung xác định những trường hợp được lên kế hoạch trước, sẽ khó khăn trong việc phát hiện các lỗi ẩn, nằm ngoài kế hoạch.}
\end{itemize}

\subsubsection{Unit testing}
Kiểm thử đơn vị là kiểm thử ở mức độ cơ bản, được thực hiện đối với từng module nhỏ trong hệ thống (hàm, lớp, phương thức). Kiểm thử đơn vị thông thường được thực hiện trực tiếp bởi đội ngũ lập trình viên mà không cần lưu lại hay quản lý ở các mức độ kiểm thử khác.

\begin{itemize}
    \item Ưu điểm
        \subitem{Giúp chúng ta chỉnh sửa và tối ưu code dễ dàng hơn và an toàn hơn, việc có unit test cho từng module sẽ đảm bảo chức năng của các module không bị ảnh hưởng bởi thay đổi vì sẽ luôn được rà soát và kiểm tra.}
        \subitem{Unit testing là bước đệm giúp chúng ta thực hiện integration test. Phát hiện lỗi sớm sẽ giúp tiết kiệm thời gian và chi phí của cả quá trình kiểm thử về sau.}
        \subitem{Hỗ trợ việc tái sử dụng testcase.}
    \item Nhược điểm
        \subitem{Không hỗ trợ phát hiện các lỗi liên quan tới việc tích hợp giữa nhiều module khác nhau hoặc việc tích hợp những logic từ ngữ cảnh vào.}
\end{itemize}

\subsubsection{Integration testing}
Kiểm thử tích hợp là loại kiểm thử phần mềm hoặc chức năng được tích hợp logic và kiểm tra theo nhóm chung với nhau. Mức độ này được thực hiện bởi tester. Đây là loại kiểm thử tập trung vào việc kiểm tra giao diện (interface) giữa các thành phần trong hệ thống với nhau.

\begin{itemize}
    \item Ưu điểm
        \subitem{Đảm bảo được tính tương thích hoạt động giữa những thành phần trong hệ thống.}
        \subitem{Giúp phát hiện được những lỗi khi các module trong hệ thống hoạt động kết hợp với nhau, việc tích hợp logic vào kiểm thử cũng sẽ hỗ trợ bao phủ toàn bộ các trường hợp có thể xảy ra trên hệ thống.}
    \item Nhược điểm
        \subitem{Chi phí dành cho kiểm thử tích hợp sẽ lớn hơn kiểm thử đơn vị.}
        \subitem{Các lỗi xuất hiện trong giai đoạn tích hợp logic giữa nhiều module có thể khó khăn để phân tích và cần sự kết hợp giữa tester và developer.}
\end{itemize}

\subsubsection{System testing}
Kiểm thử hệ thống là kiểm thử toàn bộ chức năng và giao diện của hệ thống. Sự khác nhau giữa system test và integration test là system test tập trung vào các hành vi và lỗi trên toàn bộ hệ thống, còn integration test thì chú trọng vào sự giao tiếp giữa các đơn thể hoặc đối tượng khi chúng hoạt động cùng với nhau. Chúng ta cần thông qua unit và integration test để đảm bảo mọi thành phần và sự tương tác giữa chúng hoạt động linh hoạt trước khi thực hiện system test.

\begin{itemize}
    \item Ưu điểm
        \subitem{Đảm bảo tính toàn vẹn trong hoạt động của hệ thống từ những hành vi, chức năng đến các trường hợp gây ra ngoại lệ và lỗi.}
    \item Nhược điểm
        \subitem{Chi phí để thực hiện system test sẽ nhiều hơn integration test, hơn nữa system test cần nên được thực hiện sau khi đã thông qua integration test và unit test để đảm bảo được hoạt động trên hệ thống ổn định và linh hoạt với các logic đầu vào khác nhau.}
\end{itemize}

\subsubsection{Acceptance testing}
Tương tự như system test nhưng acceptance testing thường được khách hàng trực tiếp thực hiện, mục đích là xem phần mềm có đáp ứng được yêu cầu khách hàng hay chưa.

Gồm 2 loại:
\begin{enumerate}
    \item Alpha test được thực hiện bởi các thành viên quản lý sản phẩm (không liên quan trực tiếp đến dự án), một hình thức kiểm thử nội bộ trước khi tiến hành kiểm thử Beta.
    \item Beta test là kiểm thử được thực hiện bởi người dùng cuối (thường là khách hàng), được thực hiện tại địa điểm của khách hàng, môi trường của riêng họ.
\end{enumerate}

\begin{itemize}
    \item Ưu điểm
        \subitem{Đảm bảo được những mục tiêu chính của sản phẩm đối với khách hàng, đây sẽ là bước cuối trước khi chính thức đưa sản phẩm đến tay của người dùng cuối.}
        \subitem {Người dùng cuối sẽ đưa ra được nhiều góc nhìn mà người kiểm thử, người phát triển không thấy được.}
    \item Nhược điểm
        \subitem{Acceptance test sẽ được thực hiện kiểm thử bởi người dùng cuối, vì vậy kết quả cuối cùng sẽ là thứ xoay quanh nhu cầu của họ. Điều đó đồng nghĩa sẽ có nhiều flow bị bỏ qua.}
        \subitem{Việc thực hiện acceptance test ảnh hưởng nhiều bởi cá nhân người kiểm thử, QA hay PM biết quá nhiều về cách hoạt động của dự án, khó để đảm bảo background của người kiểm thử tương ứng với một khách hàng thật sự.}
\end{itemize}

\subsection{Thực hiện kiểm thử}

Sau khi cân nhắc những ưu, nhược điểm cũng như yêu cầu về thời gian và nhân lực của các phương pháp kiểm thử, nhóm quyết định sử dụng phương pháp kiểm thử Usecase testing để kiểm thử hệ thống.

Dưới đây là tổng hợp các trường hợp kiểm thử dựa trên usecase, nội dung chi tiết được tổng hợp tại \href{https://docs.google.com/spreadsheets/d/1yDyn8-fa67zlSFjvmJKY9pEwQEnEU2gvkhm_gCw2BHo/edit#gid=0}{đây}.

\begin{table}[H]
    \centering
    \resizebox{0.8\textwidth}{!}{%
    \begin{tabular}{|p{2cm}|p{10cm}|}
    \hline
    {\textbf{STT}} & {\textbf{Tên usecase}} \\ \hline
    {1} & {Truy cập trang quản lý workspace} \\ \hline
    {2} & {Truy cập workspace} \\ \hline
    {3} & {Tạo mới workspace} \\ \hline
    {4} & {Xóa workspace} \\ \hline
    {5} & {Sử dụng bộ lọc tìm kiếm workspace} \\ \hline
    {6} & {Ghim workspace} \\ \hline
    {7} & {Thay đổi tên workspace} \\ \hline
    {8} & {Tìm kiếm workspace} \\ \hline
    {9} & {Chia sẻ workspace} \\ \hline
    {10} & {Tìm kiếm project trong workspace} \\ \hline
    {11} & {Loại thành viên khỏi workspace} \\ \hline
    {12} & {Thay đổi quyền truy cập của thành viên} \\ \hline
    {13} & {Sử dụng bộ lọc tìm kiếm thành viên} \\ \hline
    {14} & {Mời người dùng khác vào workspace} \\ \hline
    {15} & {Quản lý thành viên trong workspace} \\ \hline
    {16} & {Tìm kiếm thành viên trong workspace} \\ \hline
    {17} & {Sử dụng bộ lọc tìm kiếm yêu cầu} \\ \hline
    {18} & {Truy cập trang quản lý yêu cầu trong workspace} \\ \hline
    {19} & {Chấp nhận yêu cầu từ thành viên trong workspace} \\ \hline
    {20} & {Từ chối yêu cầu từ thành viên trong workspace} \\ \hline
    {21} & {Xóa yêu cầu từ thành viên trong workspace} \\ \hline
    {22} & {Nhận yêu cầu từ thành viên trong workspace} \\ \hline
    {23} & {Tìm kiếm yêu cầu trong workspace} \\ \hline
    {24} & {Chỉnh sửa giá trị thang đo của workspace} \\ \hline
    {25} & {Chỉnh sửa thông tin của process version} \\ \hline
    {26} & {Kích hoạt process version} \\ \hline
    {27} & {Khởi tạo process portfolio} \\ \hline
    {28} & {Chỉnh sửa thông tin của process version bị khuyết giá trị} \\ \hline
    {29} & {Tạo mới bảng khảo sát} \\ \hline
    {30} & {Đóng bảng khảo sát} \\ \hline
    {31} & {Xóa bảng khảo sát} \\ \hline
    {32} & {Cấu hình bảng khảo sát} \\ \hline
    {33} & {Cấu hình câu hỏi} \\ \hline
    {34} & {Tạo mới câu hỏi} \\ \hline
    {35} & {Xóa câu hỏi} \\ \hline
    {36} & {Thực hiện bảng khảo sát} \\ \hline
    {37} & {Quản lý kết quả bảng khảo sát} \\ \hline
    {38} & {Di chuyển vị trí câu hỏi} \\ \hline
    {39} & {Chế độ xem trước bảng khảo sát} \\ \hline
    {40} & {Công bố bảng khảo sát} \\ \hline
    {41} & {Chấp nhận lời mời vào workspace} \\ \hline
    {42} & {Xem nội dung chi tiết thông báo} \\ \hline
    {43} & {Xem thông báo} \\ \hline
    {44} & {Từ chối lời mời vào workspace} \\ \hline
    {45} & {Lọc thông báo} \\ \hline
    {46} & {Ghim thông báo} \\ \hline
    {47} & {Nhận thông báo từ hệ thống} \\ \hline
    {48} & {Tìm kiếm thông báo} \\ \hline
    {49} & {Xóa thông báo} \\ \hline
    \end{tabular}%
    }
    \caption{Tổng hợp các trường hợp kiểm thử usecase}
\end{table}