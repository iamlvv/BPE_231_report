\subsubsection{Chỉnh sửa thông tin thang đo của workspace}

\begin{table}[H]
    \def\arraystretch{2}%
    \centering
    \resizebox{\textwidth}{!}{%
        \begin{tabular}{|p{3cm}|p{1.5cm}|p{0.5cm}|p{4cm}|p{4cm}|>{\centering\arraybackslash}p{1cm}|}
            \hline
            \multicolumn{1}{|c|}{\textbf{Test case}} &
            \multicolumn{1}{c|}{\textbf{Flow}} &
            \multicolumn{1}{c|}{\textbf{STT}} &
            \multicolumn{1}{c|}{\textbf{Mô tả}} &
            \multicolumn{1}{c|}{\textbf{Mong đợi}} &
            \multicolumn{1}{c|}{\textbf{Kết quả}} \\ \hline
            \multirow{5}{*}{\begin{tabular}[c]{@{}l@{}}Chỉnh sửa thông \\ tin thang đo của\\ workspace\end{tabular}} &
            \multirow{4}{*}{\begin{tabular}[c]{@{}l@{}}Normal \\ flow\end{tabular}} &
            \multicolumn{1}{c|}{\centering 1} &
            Người dùng truy cập vào giao diện Quản lý workspace (Workspace management). &
            Hệ thống điều hướng tới giao diện quản lý workspace, mặc định ở trang quản lý thành viên trong workspace. &
            Đạt \\ \cline{3-6} 
            &
            &
            \multicolumn{1}{c|}{\centering 2} &
            Người dùng chọn “Process portfolio” trên sidebar. &
            Hệ thống điều hướng tới giao diện quản lý process portfolio. &
            Đạt \\ \cline{3-6} 
            &
            &
            \multicolumn{1}{c|}{\centering 3} &
            Người dùng chọn nút “Performance level”. &
            Hệ thống mở modal “Configure performance level”. Giá trị mặc định được cài đặt là 0 cho các trường thông tin. &
            Đạt \\ \cline{3-6} 
            &
            &
            \multicolumn{1}{c|}{\centering 4} &
            Người dùng nhập các trường thông tin và chọn “Save” để lưu. &
            Hệ thống lưu thông tin về thang đo của workspace. &
            Đạt \\ \hline
        \end{tabular}
    }
    \caption{Test case Chỉnh sửa thông tin thang đo của workspace}
\end{table}

\subsubsection{Chỉnh sửa thông tin thang đo của workspace: Thay đổi giá trị bất kỳ với kiểu dữ liệu chữ cái/ký tự đặc biệt} 

\begin{table}[H]
    \def\arraystretch{2}%
    \centering
    \resizebox{\textwidth}{!}{%
        \begin{tabular}{|p{3cm}|p{1.5cm}|p{0.5cm}|p{4cm}|p{4cm}|>{\centering\arraybackslash}p{1cm}|}
            \hline
            \multicolumn{1}{|c|}{\textbf{Test case}} &
            \multicolumn{1}{c|}{\textbf{Flow}} &
            \multicolumn{1}{c|}{\textbf{STT}} &
            \multicolumn{1}{c|}{\textbf{Mô tả}} &
            \multicolumn{1}{c|}{\textbf{Mong đợi}} &
            \multicolumn{1}{c|}{\textbf{Kết quả}} \\ \hline
            \multirow{5}{*}{\begin{tabular}[c]{@{}l@{}}Chỉnh sửa thông \\ tin thang đo của \\ workspace:\\ Thay đổi giá trị \\ bất kỳ với kiểu \\ dữ liệu chữ cái/ký \\ tự đặc biệt\end{tabular}} &
            \multirow{5}{*}{Exception} &
            \multicolumn{1}{c|}{1} &
            Người dùng truy cập vào giao diện Quản lý workspace (Workspace management). &
            Hệ thống điều hướng tới giao diện quản lý workspace, mặc định ở trang quản lý thành viên trong workspace. &
            Đạt \\ \cline{3-6} 
            &
            &
            \multicolumn{1}{c|}{2} &
            Người dùng chọn “Process portfolio” trên sidebar. &
            Hệ thống điều hướng tới giao diện quản lý process portfolio. &
            Đạt \\ \cline{3-6} 
            &
            &
            \multicolumn{1}{c|}{3} &
            Người dùng chọn nút “Performance level”. &
            Hệ thống mở modal “Configure performance level”. &
            Đạt \\ \cline{3-6} 
            &
            &
            \multicolumn{1}{c|}{4} &
            Người dùng nhập giá trị với kiểu dữ liệu chữ cái (Ví dụ: “abc”) vào trường thông tin bất kỳ và chọn nút “Save”. &
            Hệ thống hiển thị cảnh báo “Data type is not supported”. &
            Đạt \\ \hline
        \end{tabular}
    }
    \caption{Test case Chỉnh sửa thông tin thang đo của workspace: Thay đổi giá trị bất kỳ với kiểu dữ liệu chữ cái/ký tự đặc biệt}
\end{table}

\subsubsection{Chỉnh sửa thông tin thang đo của workspace: Người dùng thực hiện thay đổi giá trị bất kỳ với số âm} 

\begin{table}[H]
    \def\arraystretch{2}%
    \centering
    \resizebox{\textwidth}{!}{%
        \begin{tabular}{|p{3cm}|p{1.5cm}|p{0.5cm}|p{4cm}|p{4cm}|>{\centering\arraybackslash}p{1cm}|}
            \hline
            \multicolumn{1}{|c|}{\textbf{Test case}} &
            \multicolumn{1}{c|}{\textbf{Flow}} &
            \multicolumn{1}{c|}{\textbf{STT}} &
            \multicolumn{1}{c|}{\textbf{Mô tả}} &
            \multicolumn{1}{c|}{\textbf{Mong đợi}} &
            \multicolumn{1}{c|}{\textbf{Kết quả}} \\ \hline
            \multirow{5}{*}{\begin{tabular}[c]{@{}l@{}}Chỉnh sửa thông \\ tin thang đo \\ của workspace: \\ Người dùng thực \\ hiện thay đổi giá \\ trị bất kỳ với số âm\end{tabular}} &
            \multirow{5}{*}{Exception} &
            \multicolumn{1}{c|}{1} &
            Người dùng truy cập vào giao diện Quản lý workspace (Workspace management). &
            Hệ thống điều hướng tới giao diện quản lý workspace, mặc định ở trang quản lý thành viên trong workspace. &
            Đạt \\ \cline{3-6} 
            &
            &
            \multicolumn{1}{c|}{2} &
            Người dùng chọn “Process portfolio” trên sidebar. &
            Hệ thống điều hướng tới giao diện quản lý process portfolio. &
            Đạt \\ \cline{3-6} 
            &
            &
            \multicolumn{1}{c|}{3} &
            Người dùng chọn nút “Performance level”. &
            Hệ thống mở modal “Configure performance level”. &
            Đạt \\ \cline{3-6} 
            &
            &
            \multicolumn{1}{c|}{4} &
            Người dùng nhập giá trị là số âm (Ví dụ: -10) vào trường thông tin bất kỳ và chọn nút “Save”. &
            Hệ thống mặc định đặt giá trị về 0, lưu thông tin thành công và đóng modal. &
            Đạt \\ \hline
        \end{tabular}
    }
    \caption{Test case Chỉnh sửa thông tin thang đo của workspace: Người dùng thực hiện thay đổi giá trị bất kỳ với số âm}
\end{table}

\subsubsection{Chỉnh sửa thông tin thang đo của workspace: Người dùng thực hiện thay đổi giá trị cycle time/cost với kiểu dữ liệu số thập phân}

\begin{table}[H]
    \def\arraystretch{2}%
    \centering
    \resizebox{\textwidth}{!}{%
        \begin{tabular}{|p{3cm}|p{1.5cm}|p{0.5cm}|p{4cm}|p{4cm}|>{\centering\arraybackslash}p{1cm}|}
            \hline
            \multicolumn{1}{|c|}{\textbf{Test case}} &
            \multicolumn{1}{c|}{\textbf{Flow}} &
            \multicolumn{1}{c|}{\textbf{STT}} &
            \multicolumn{1}{c|}{\textbf{Mô tả}} &
            \multicolumn{1}{c|}{\textbf{Mong đợi}} &
            \multicolumn{1}{c|}{\textbf{Kết quả}} \\ \hline
            \multirow{5}{*}{\begin{tabular}[c]{@{}l@{}}Chỉnh sửa thông \\ tin thang đo \\ của workspace: \\ Người dùng thực \\ hiện thay đổi \\ giá trị cycle \\ time/cost với \\ kiểu dữ liệu \\ số thập phân\end{tabular}} &
            \multirow{5}{*}{Exception} &
            \multicolumn{1}{c|}{1} &
            Người dùng truy cập vào giao diện Quản lý workspace (Workspace management). &
            Hệ thống điều hướng tới giao diện quản lý workspace, mặc định ở trang quản lý thành viên trong workspace. &
            Đạt \\ \cline{3-6} 
            &
            &
            \multicolumn{1}{c|}{2} &
            Người dùng chọn “Process portfolio” trên sidebar. &
            Hệ thống điều hướng tới giao diện quản lý process portfolio. &
            Đạt \\ \cline{3-6} 
            &
            &
            \multicolumn{1}{c|}{3} &
            Người dùng chọn nút “Performance level”. &
            Hệ thống mở modal “Configure performance level”. &
            Đạt \\ \cline{3-6} 
            &
            &
            \multicolumn{1}{c|}{4} &
            Người dùng nhập giá trị cho nhóm cycle time/cost là số thập phân (Ví dụ: 0.1) và chọn nút “Save”. &
            Hệ thống mặc định làm tròn giá trị về số nguyên gần nhất, lưu thông tin thành công và đóng modal. &
            \multicolumn{1}{c|}{Đạt} \\ \hline
        \end{tabular}
    }
    \caption{Test case Chỉnh sửa thông tin thang đo của workspace: Người dùng thực hiện thay đổi giá trị cycle time/cost với kiểu dữ liệu số thập phân}
\end{table}

\subsubsection{Chỉnh sửa thông tin thang đo của workspace: Người dùng thực hiện thay đổi giá trị quality/flexibility với số nằm ngoài khoảng [0, 1]}

\begin{table}[H]
    \def\arraystretch{2}%
    \centering
    \resizebox{\textwidth}{!}{%
        \begin{tabular}{|p{3cm}|p{1.5cm}|p{0.5cm}|p{4cm}|p{4cm}|>{\centering\arraybackslash}p{1cm}|}
            \hline
            \multicolumn{1}{|c|}{\textbf{Test case}} &
            \multicolumn{1}{c|}{\textbf{Flow}} &
            \multicolumn{1}{c|}{\textbf{STT}} &
            \multicolumn{1}{c|}{\textbf{Mô tả}} &
            \multicolumn{1}{c|}{\textbf{Mong đợi}} &
            \multicolumn{1}{c|}{\textbf{Kết quả}} \\ \hline
            \multirow{5}{*}{\begin{tabular}[c]{@{}l@{}}Chỉnh sửa thông \\ tin thang đo \\ của workspace: \\ Người dùng thực \\ hiện thay đổi \\ giá trị quality/ \\ flexibility với số \\ nằm ngoài khoảng \\ {[}0, 1{]}\end{tabular}} &
            \multirow{5}{*}{Exception} &
            \multicolumn{1}{c|}{1} &
            Người dùng truy cập vào giao diện Quản lý workspace (Workspace management). &
            Hệ thống điều hướng tới giao diện quản lý workspace, mặc định ở trang quản lý thành viên trong workspace. &
            Đạt \\ \cline{3-6} 
            &
            &
            \multicolumn{1}{c|}{2} &
            Người dùng chọn “Process portfolio” trên sidebar. &
            Hệ thống điều hướng tới giao diện quản lý process portfolio. &
            Đạt \\ \cline{3-6} 
            &
            &
            \multicolumn{1}{c|}{3} &
            Người dùng chọn nút “Performance level”. &
            Hệ thống mở modal “Configure performance level”. &
            Đạt \\ \cline{3-6} 
            &
            &
            \multicolumn{1}{c|}{4} &
            Người dùng nhập giá trị cho nhóm quality/flexibility nằm ngoài khoảng [0,1] (Ví dụ: 2) và chọn nút “Save”. &
            Hệ thống mặc định làm tròn về 1 nếu giá trị nhập vào lớn hơn 1 hoặc 0 nếu giá trị nhập vào nhỏ hơn 0, lưu thông tin thành công và đóng modal. &
            \multicolumn{1}{c|}{Đạt} \\ \hline
        \end{tabular}
    }
    \caption{Test case Chỉnh sửa thông tin thang đo của workspace: Người dùng thực hiện thay đổi giá trị quality/flexibility với số nằm ngoài khoảng [0, 1]}
\end{table}

\subsubsection{Chỉnh sửa thông tin thang đo của workspace: Người dùng thực hiện thay đổi giá trị cycle time/cost với target > worst}

\begin{table}[H]
    \def\arraystretch{2}%
    \centering
    \resizebox{\textwidth}{!}{%
        \begin{tabular}{|p{3cm}|p{1.5cm}|p{0.5cm}|p{4cm}|p{4cm}|>{\centering\arraybackslash}p{1cm}|}
            \hline
            \multicolumn{1}{|c|}{\textbf{Test case}} &
            \multicolumn{1}{c|}{\textbf{Flow}} &
            \multicolumn{1}{c|}{\textbf{STT}} &
            \multicolumn{1}{c|}{\textbf{Mô tả}} &
            \multicolumn{1}{c|}{\textbf{Mong đợi}} &
            \multicolumn{1}{c|}{\textbf{Kết quả}} \\ \hline
            \multirow{5}{*}{\begin{tabular}[c]{@{}l@{}}Chỉnh sửa thông \\ tin thang đo \\ của workspace: \\ Người dùng thực \\ hiện thay đổi \\ giá trị cycle \\ time/cost với \\ target > worst\end{tabular}} &
            \multirow{5}{*}{Exception} &
            \multicolumn{1}{c|}{1} &
            Người dùng truy cập vào giao diện Quản lý workspace (Workspace management). &
            Hệ thống điều hướng tới giao diện quản lý workspace, mặc định ở trang quản lý thành viên trong workspace. &
            Đạt \\ \cline{3-6} 
            &
            &
            \multicolumn{1}{c|}{2} &
            Người dùng chọn “Process portfolio” trên sidebar. &
            Hệ thống điều hướng tới giao diện quản lý process portfolio. &
            Đạt \\ \cline{3-6} 
            &
            &
            \multicolumn{1}{c|}{3} &
            Người dùng chọn nút “Performance level”. &
            Hệ thống mở modal “Configure performance level”. &
            Đạt \\ \cline{3-6} 
            &
            &
            \multicolumn{1}{c|}{4} &
            Người dùng nhập giá trị cho nhóm cycle time/cost với target value > worst value (Ví dụ: target cycle time = 100, worst cycle time = 10) và chọn nút “Save”. &
            Hệ thống cảnh báo “Targeted cycle time must be less than worst cycle time”. &
            \multicolumn{1}{c|}{Đạt} \\ \hline
        \end{tabular}
    }
    \caption{Test case Chỉnh sửa thông tin thang đo của workspace: Người dùng thực hiện thay đổi giá trị cycle time/cost với target > worst}
\end{table}

\subsubsection{Chỉnh sửa thông tin thang đo của workspace: Người dùng thực hiện thay đổi giá trị quality/flexibility với target < worst}

\begin{table}[H]
    \def\arraystretch{2}%
    \centering
    \resizebox{\textwidth}{!}{%
        \begin{tabular}{|p{3cm}|p{1.5cm}|p{0.5cm}|p{4cm}|p{4cm}|>{\centering\arraybackslash}p{1cm}|}
            \hline
            \multicolumn{1}{|c|}{\textbf{Test case}} &
            \multicolumn{1}{c|}{\textbf{Flow}} &
            \multicolumn{1}{c|}{\textbf{STT}} &
            \multicolumn{1}{c|}{\textbf{Mô tả}} &
            \multicolumn{1}{c|}{\textbf{Mong đợi}} &
            \multicolumn{1}{c|}{\textbf{Kết quả}} \\ \hline
            \multirow{5}{*}{\begin{tabular}[c]{@{}l@{}}Chỉnh sửa thông \\ tin thang đo \\ của workspace: \\ Người dùng thực \\ hiện thay đổi \\ giá trị quality/ \\ flexibility với \\ target < worst\end{tabular}} &
            \multirow{5}{*}{Exception} &
            \multicolumn{1}{c|}{1} &
            Người dùng truy cập vào giao diện Quản lý workspace (Workspace management). &
            Hệ thống điều hướng tới giao diện quản lý workspace, mặc định ở trang quản lý thành viên trong workspace. &
            Đạt \\ \cline{3-6} 
            &
            &
            \multicolumn{1}{c|}{2} &
            Người dùng chọn “Process portfolio” trên sidebar. &
            Hệ thống điều hướng tới giao diện quản lý process portfolio. &
            Đạt \\ \cline{3-6} 
            &
            &
            \multicolumn{1}{c|}{3} &
            Người dùng chọn nút “Performance level”. &
            Hệ thống mở modal “Configure performance level”. &
            Đạt \\ \cline{3-6} 
            &
            &
            \multicolumn{1}{c|}{4} &
            Người dùng nhập giá trị cho nhóm quality/flexibility với target value < worst value (Ví dụ: target quality = 0.5, worst quality = 0.8) và chọn nút “Save”. &
            Hệ thống cảnh báo “Worst quality must be less than targeted quality” &
            \multicolumn{1}{c|}{Đạt} \\ \hline
        \end{tabular}
    }
    \caption{Test case Chỉnh sửa thông tin thang đo của workspace: Người dùng thực hiện thay đổi giá trị quality/flexibility với target < worst}
\end{table}
