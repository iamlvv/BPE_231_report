%   Vai trò của hệ thống trong tổ chức/doanh nghiệp là gì? Ý nghĩa của vai trò đó?

Với mục tiêu phát triển hiện tại của hệ thống, BPSky sẽ trở thành công cụ hỗ trợ đắc lực của nhà quản lý quy trình nghiệp vụ trong việc tự động hóa quá trình đánh giá quy trình nghiệp vụ dựa trên mô hình \acrfull*{bpmn}.

Bên cạnh đó, hệ thống nhắm tới mục tiêu mở rộng phạm vi quản lý mô hình quy trình nghiệp vụ; từ quản lý mô hình trong không gian làm việc cá nhân, mở rộng thành không gian làm việc chung có thể được truy cập và chia sẻ giữa nhiều người dùng; điều này giúp nhà quản lý tiếp cận hệ thống và thực hiện công việc quản lý dễ dàng hơn.

Ngoài ra, hệ thống còn tích hợp khảo sát vào tính năng đánh giá chất lượng quy trình nghiệp vụ nhằm thu thập trải nghiệm của khách hàng và người tham gia vào các tác vụ trong quy trình. Điều này nhằm đem lại góc nhìn đa chiều hơn trong việc đánh giá chất lượng quy trình nghiệp vụ; không chỉ từ góc nhìn của hệ thống và người quản lý quy trình, mà còn dựa trên góc nhìn của khách hàng trải nghiệm sản phẩm/dịch vụ của doanh nghiệp và những cá nhân tham gia thực thi quy trình nghiệp vụ. Yếu tố này sẽ giúp đánh giá chất lượng quy trình nghiệp vụ, vốn đã là một độ đo có phần trừu tượng và bị ảnh hưởng bởi nhiều yếu tố, theo hướng khách quan và có độ tin cậy cao hơn.

Hơn nữa, hệ thống hỗ trợ người dùng có được góc nhìn tổng quát đối với các quy trình đang hoạt động trong không gian làm việc. Điều này đồng thời liên quan tới việc giám sát và đưa ra các quyết định đối với quy trình trong tổ chức.