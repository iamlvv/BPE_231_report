%   Vai trò của hệ thống trong tổ chức/doanh nghiệp là gì? Ý nghĩa của vai trò đó?

Với mục tiêu phát triển hiện tại của hệ thống, BKSky sẽ trở thành công cụ hỗ
trợ đắc lực của nhà quản lý quy trình nghiệp vụ trong việc tự động hóa quá
trình đánh giá quy trình nghiệp vụ dựa trên mô hình BPMN (Business process
model notation).

Bên cạnh đó, hệ thống nhắm tới mục tiêu mở rộng phạm vi quản lý mô hình quy
trình nghiệp vụ; từ quản lý mô hình đơn lẻ, mở rộng ra thành quản lý một nhóm
những mô hình; điều này giúp nhà quản lý tiếp cận hệ thống và thực hiện công
việc quản lý dễ dàng hơn.

Ngoài ra, hệ thống còn phát triển bảng khảo sát chất lượng quy trình nghiệp vụ
nhằm thu thập trải nghiệm của khách hàng và người tham gia vào các tác vụ trong
quy trình. Điều này nhằm đem lại góc nhìn đa chiều hơn trong việc đánh giá chất
lượng quy trình nghiệp vụ; không chỉ từ góc nhìn của hệ thống và người quản lý
quy trình, mà còn dựa trên góc nhìn của khách hàng trải nghiệm sản phẩm/dịch vụ
của doanh nghiệp và những cá nhân tham gia thực thi quy trình nghiệp vụ. Yếu tố
này sẽ giúp đánh giá chất lượng quy trình nghiệp vụ, vốn đã là một đánh giá có
phần trừu tượng và bị ảnh hưởng bởi nhiều yếu tố, theo hướng khách quan và có
độ tin cậy cao hơn.

Hơn nữa, hệ thống hỗ trợ nhà quản lý quy trình nghiệp vụ đưa ra quyết định giám
sát và cải tiến ở những quy trình đem lại tổn thất về hiệu suất hoặc tính khả
thi còn hạn chế. Điều này hỗ trợ tổ chức và doanh nghiệp đưa ra những hành động
cần thiết đối với quy trình nghiệp vụ, không chỉ ở giai đoạn đánh giá mà còn
trong những giai đoạn tái thiết kế, tự động hóa,...