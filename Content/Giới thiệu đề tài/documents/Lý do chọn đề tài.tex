Ngay từ môi trường đại học, nhu cầu nắm bắt được quy trình nghiệp vụ là không thể thiếu đối với mọi sinh viên, từ những quy trình cung cấp dịch vụ ăn uống cho đến những quy trình hoạt động của nghiệp vụ ở phòng đào tạo. 

Thực tế hơn trong những công ty làm việc theo hướng lấy quy trình hóa nghiệp vụ làm trung tâm (process-centered way), nhiệm vụ quản lý quy trình nghiệp vụ đóng vai trò không thể thiếu trong việc giám sát và điều phối hoạt động của tổ chức. Nhưng một tổ chức lại có nhiều nhóm quy trình nghiệp vụ với độ ưu tiên khác nhau, điều này đặt ra nhu cầu mở rộng phạm vi quản lý và có được cái nhìn tổng quan về toàn bộ quy trình của người quản lý quy trình nghiệp vụ thay vì tập trung vào giám sát một quy trình nghiệp vụ đơn lẻ.

Từ đó, khi đã có được một công cụ hỗ trợ quản lý đa quy trình trong tổ chức, quá trình đánh giá mô hình quy trình nghiệp vụ (BPM - Business process model) sẽ có nhiệm vụ đề ra được độ ưu tiên trong một nhóm quy trình. Độ ưu tiên sẽ đóng vai trò là thang đo, giúp tổ chức nhận biết được đối tượng nào tạo ra tổn thất hoặc có tiềm năng nhưng chưa được quan tâm đúng cách để có thể tập trung tài nguyên và nhân lực cho mục tiêu vận hành quy trình năng suất và hiệu quả hơn.

Chính vì những nhu cầu trên nên chúng tôi quyết định kế thừa hệ thống đánh giá mô hình quy trình nghiệp vụ BKSky 1.0 để phát triển hệ thống BKSky 2.0. Mục tiêu của hệ thống là mở rộng phạm vi ảnh hưởng từ đánh giá, quản lý đơn quy trình thành đánh giá, quản lý đa quy trình; hỗ trợ trực quan hóa mối quan hệ ưu tiên giữa những quy trình trong tổ chức thông qua những tiêu chí về độ quan trọng chiến lược, hiệu suất hoạt động và tính khả thi của quy trình nghiệp vụ.