Cấu trúc của đồ án gồm:
\begin{itemize}
      \item \textbf{Chương 1:} Giới thiệu đề tài, mục đích chọn đề tài, ý nghĩa và phạm vi đề tài, cấu trúc đồ án và kế hoạch phân chia công việc chi tiết.
      \item \textbf{Chương 2:} Cơ sở lý thuyết cho việc đánh giá mô hình \acrshort{bpmn}, lý thuyết thiết kế và tính toán kết quả khảo sát, lý thuyết khởi tạo danh mục quy trình.
      \item \textbf{Chương 3:} Phân tích và thiết kế hệ thống
          \begin{itemize}
            \item Giới thiệu hệ thống: Đưa ra những yêu cầu người dùng cần để thực hiện hệ thống,
                  bao gồm các yêu cầu chức năng và phi chức năng.
            \item Use-case: lược đồ use-case của toàn bộ hệ thống và đặc tả use-case.
            \item Sơ đồ hoạt động: bao gồm sơ đồ hoạt động của các tính năng chính trong hệ
                  thống.
            \item Kiến trúc phần mềm: mô tả kiến trúc hệ thống, so sánh kiến trúc hệ thống với
                  các dạng kiến trúc khác nhau.
            \item Sơ đồ lớp: vẽ và mô tả lớp theo mô hình phân lớp của hệ thống.
            \item Thiết kế cơ sở dữ liệu: lựa chọn cơ sở dữ liệu và vẽ \acrfull*{erd}, đặc tả các thực thể,
                  mối quan hệ trong cơ sở dữ liệu.
          \end{itemize}
      \item \textbf{Chương 4:} Hiện thực hệ thống
          \begin{itemize}
            \item Công nghệ sử dụng: giới thiệu những công nghệ được sử dụng để hiện thực hệ
                  thống bao gồm front-end, back-end, database.
            \item Hiện thực giao diện người dùng: Giới thiệu các trang giao diện chính của phần
                  mềm.
            \item Triển khai hệ thống
          \end{itemize}
      \item \textbf{Chương 5:} Kiểm thử và đánh giá hệ thống
          \begin{itemize}
            \item Kiểm thử hệ thống: Sử dụng kỹ thuật use-case testing để kiểm thử các nhóm chức năng được hiện thực trong hệ thống.
            \item Case study: Ứng dụng hệ thống vào quy trình của một doanh nghiệp thực tế.
            \item Kiểm chứng: Kiểm chứng lại kết quả của chức năng mà hệ thống cung cấp so với tính toán lý thuyết.
          \end{itemize}
      \item \textbf{Chương 6:} Kết luận
          \begin{itemize}
            \item Tổng kết kết quả đạt được, nhìn nhận những hạn chế còn tồn đọng.
            \item Hướng phát triển: Định hướng phát triển hệ thống trong tương lai.
          \end{itemize}
\end{itemize}