Cấu trúc của đồ án gồm:
\begin{itemize}
    \item \textbf{Chương 1:} Giới thiệu đề tài, mục đích chọn đề tài, ý nghĩa và phạm vi đề tài.
    \item \textbf{Chương 2:} Cơ sở lý thuyết cho việc đánh giá mô hình \acrshort{bpmn}.
    \item \textbf{Chương 3:} Phân tích và thiết kế hệ thống
          \begin{itemize}
              \item Giới thiệu hệ thống: Đưa ra những yêu cầu người dùng cần để thực hiện hệ thống,
                    bao gồm các yêu cầu chức năng và phi chức năng.
              \item Kiến trúc phần mềm: mô tả kiến trúc hệ thống, so sánh kiến trúc hệ thống với
                    các dạng kiến trúc khác nhau.
              \item Use-case: lược đồ use-case của toàn bộ hệ thống và đặc tả use-case.
              \item Sơ đồ hoạt động: bao gồm sơ đồ hoạt động của các tính năng chính trong hệ
                    thống.
              \item Sơ đồ lớp: vẽ và mô tả lớp theo mô hình phân lớp của hệ thống.
              \item Thiết kế cơ sở dữ liệu: lựa chọn cơ sở dữ liệu và vẽ \acrfull*{erd}, đặc tả các thực thể,
                    mối quan hệ trong cơ sở dữ liệu.
          \end{itemize}
    \item \textbf{Chương 4:} Hiện thực hệ thống
          \begin{itemize}
              \item Công nghệ sử dụng: giới thiệu những công nghệ được sử dụng để hiện thực hệ
                    thống bao gồm front-end, back-end, database.
              \item Hiện thực giao diện người dùng: Giới thiệu các trang giao diện chính của phần
                    mềm.
          \end{itemize}
    \item \textbf{Chương 5:} Kết luận
          \begin{itemize}
              \item Tổng kết kết quả đạt được, nhìn nhận những hạn chế còn tồn đọng.
              \item Hướng phát triển: Định hướng phát triển hệ thống cho giai đoạn luận văn.
          \end{itemize}
\end{itemize}