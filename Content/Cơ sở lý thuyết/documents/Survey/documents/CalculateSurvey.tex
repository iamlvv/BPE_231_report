\subsection{Tính toán kết quả khảo sát}
Như đã đề cập, bài khảo sát sẽ có những câu hỏi để tính toán các giá trị để đánh giá mức độ hài lòng của người dùng đối với một quy trình nghiệp vụ. Các độ đo ở những thang điểm khác nhau, cụ thể CES và CSAT tính trên thang điểm 1 - 7, NPS trên thang điểm 0 - 10. Vì thế, chúng tôi nhận thấy cần phải chuẩn hóa các độ đo về một miền giá trị duy nhất. 
\par
Hiện nay có nhiều phương pháp sử dụng để chuẩn hoá dữ liệu, và \emph{Min-Max} là một trong số đó. Chuẩn hoá \emph{Min-Max} biểu diễn sự biến đổi tuyến tính trên dữ liệu ban đầu. Biết rằng $min_a$ và $max_a$ lần lượt là giá trị nhỏ nhất và giá trị lớn nhất của thuộc tính A. Phương pháp chuẩn hoá \emph{Min-Max} sẽ ánh xạ một giá trị $v$ của $A$ thành $v'$ trong khoảng $(new-min_a,$ $new-max_a)$ bằng việc tính toán như công thức dưới đây:
\[ v' = (\frac{(v - min_a}{max_a - min_a}) \times ((new-max_a) - (new-min_a)) + new-min_a\]

Ở đây chúng tôi chọn một miền giá trị phổ biến là [0, 1], và đưa ra công thức chuẩn hoá cụ thể cho từng độ đo như sau, vận dụng phương pháp chuẩn hoá \emph{Min-Max} được trình bày ở trên:

\textbf{Độ đo CES:}
\[ \text{\emph{Normalized CES}} = (\frac{CES - CES_{Min}}{CES_{Max} - CES_{Min}}) \times 100\]
\par
Trong đó, \emph{CES} là giá trị \emph{CES} tính được từ kết quả khảo sát, $CES_{Min}$ và $CES_{Max}$ lần lượt là giá trị nhỏ nhất và giá trị lớn nhất của điểm \emph{CES}, hay cận dưới và cận trên của thang điểm của độ đo \emph{CES}. Ví dụ, trên miền giá trị của điểm CES là [0, 1], ta có $CES_{Min} = 0$ và $CES_{Max} = 1$

\textbf{Độ đo CSAT:}
\[ \text{\emph{Normalized CSAT}} = (\frac{CSAT - CSAT_{Min}}{CSAT_{Max} - CSAT_{Min}}) \times 100\]
\par
Trong đó, \emph{CSAT} là giá trị \emph{CSAT} tính được từ kết quả khảo sát, $CSAT_{Min}$ và $CSAT_{Max}$ lần lượt là giá trị nhỏ nhất và giá trị lớn nhất của điểm \emph{CSAT}, hay cận dưới và cận trên của thang điểm của độ đo \emph{CSAT}. Ví dụ, trên miền giá trị của điểm CSAT là [0, 1], ta có $CSAT_{Min} = 0$ và $CSAT_{Max} = 1$

\textbf{Độ đo NPS:}
\[ \text{\emph{Normalized NPS}} = (\frac{NPS - NPS_{Min}}{NPS_{Max} - NPS_{Min}}) \times 100\]
\par
Trong đó, \emph{NPS} là giá trị \emph{NPS} tính được từ kết quả khảo sát, $NPS_{Min}$ và $NPS_{Max}$ lần lượt là giá trị nhỏ nhất và giá trị lớn nhất của điểm \emph{NPS}, hay cận dưới và cận trên của thang điểm của độ đo \emph{NPS}. Ví dụ, trên miền giá trị của điểm NPS là [-1, 1], ta có $NPS_{Min} = -1$ và $NPS_{Max} = 1$
\par
Tuỳ vào mỗi loại quy trình nghiệp vụ mà người thiết kế sẽ ưu tiên xem xét độ đo nào chiếm tỉ trọng lớn hơn trong giá trị chung của bảng khảo sát. Vì thế, ứng dụng phương pháp MUSA đã được trình bày ở trên, chúng tôi đề xuất công thức tính giá trị chung như sau:
\[ \text{Survey Score} = (w_{CES} \times \text{Normalized CES}) + (w_{NPS} \times \text{Normalized NPS}) + (w_{CSAT} \times \text{Normalized CSAT})\]
\[ w_{CES} + w_{CSAT} + w_{NPS} = 1\]
\par
Trong đó, $w_{CES}$, $w_{CSAT}$, $w_{NPS}$ lần lượt là trọng số của giá trị CES, NPS và CSAT đã chuẩn hóa. Trọng số này có thể được quy định bởi người thiết kế bảng khảo sát, tuỳ thuộc vào độ ưu tiên của họ đối với độ đo nào cho bảng khảo sát hay quy trình nghiệp vụ.

