Đánh giá \emph{external quality} đồng nghĩa với việc đánh giá được sự hài lòng của người dùng hoặc khách hàng đối với một quy trình nghiệp vụ. Để làm được như vậy, cần phải thu thập những ý kiến của người sử dụng hoặc theo dõi và tự động đánh giá quá trình người dùng thực thi quy trình nghiệp vụ. Ngày nay, có một số công cụ dưới dạng các tiện ích bổ sung (add-on extensions) trên các trình duyệt web, cho phép tích hợp vào hệ thống để theo dõi các luồng thực thi của khách hàng trên một quy trình cụ thể, chẳng hạn như khách hàng có hoàn thành phiên hoạt động của mình hay không, khách hàng có rơi vào những trường hợp ngoại lệ không, có khách hàng nào dừng quy trình giữa chừng hay không.
Ứng dụng các công cụ trên vào đánh giá chất lượng bên ngoài của quy trình có thể đảm bảo tính chính xác, và được thực thi một cách tự động dựa trên việc theo dõi hành vi của người dùng, không tốn nhiều thời gian, công sức của người dùng . Tuy nhiên, không phải quy trình nghiệp vụ nào cũng có thể được tích hợp công cụ trên. Một số quy trình nghiệp vụ cần được thực thi ngoài thực tế, không phải trên các trang web hay hệ thống máy tính, dẫn tới không thể ứng dụng các công cụ theo dõi hành vi người dùng vào những quy trình nghiệp vụ này. Chúng tôi nhận thấy rằng, đối với những loại quy trình này nói riêng, hay cả những quy trình khác nói chung, đều có thể thu thập được sự hài lòng và trải nghiệm của người dùng thông qua các cuộc khảo sát (survey).
\par
Khảo sát đã được ứng dụng từ lâu để lấy ý kiến khách hàng về một dịch vụ, sản phẩm,... hay thậm chí là cả một tổ chức, một doanh nghiệp. Ngày nay, khảo sát vẫn là một công cụ phổ biến và hiệu quả để thu thập thông tin từ người dùng, cho nên ta có thể sử dụng khảo sát, với các câu hỏi khai thác hợp lý, có thể đánh giá được thái độ của khách hàng đối với quy trình nghiệp vụ cụ thể như thế nào, từ đó có thể rút ra được \emph{external quality} của quy trình. Khảo sát có thể được thực thi trên nhiều nền tảng khác nhau với nhiều hình thức khác nhau, chẳng hạn như có thể khảo sát khách hàng thông qua chatbot - là một phần mềm ứng dụng trí tuệ nhân tạo để mô phỏng lại các cuộc trò chuyện với người dùng, thông qua cuộc trò chuyện đó có thể lấy được thông tin từ người dùng. Hoặc có thể là một bảng khảo sát trực tuyến với các câu hỏi đa dạng hình thức được thiết kế sẵn để người dùng trả lời và gửi phản hồi về quy trình đó. Khảo sát trực tuyến có một số lợi ích như sau:
\begin{itemize}
    \item Người dùng có thể thực hiện khảo sát bất cứ khi nào họ muốn.
    \item Người dùng có thể dành nhiều thời gian hơn để trả lời các câu hỏi, nói cách khác, họ không bị ràng buộc thời gian.
    \item Khảo sát trực tuyến cho phép người thiết kế đưa ra nhiều hình thức câu hỏi khác nhau để đánh giá nhiều khía cạnh khác nhau của vấn đề cần được khảo sát (câu hỏi hai đáp án, nhiều đáp án, câu hỏi dạng thang điểm,...).
\end{itemize}
\par
Vì một số ưu điểm nêu trên, chúng tôi quyết định lựa chọn khảo sát trực tuyến như là phương tiện chính để đánh giá sự hài lòng của khách hàng đối với quy trình nghiệp vụ.