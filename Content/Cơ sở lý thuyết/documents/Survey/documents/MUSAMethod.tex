\subsection{Phương pháp \acrshort*{musa}}
Một vấn đề đặt ra, đó bảng khảo sát đo đạc giá trị CES, CSAT, NPS không chỉ trên một yếu tố của quy trình, mà còn trên nhiều yếu tố khác, chẳng hạn như khả năng xử lý các ngoại lệ và độ linh hoạt của quy trình, cần phải tổng hợp các giá trị đó thành một giá trị đo lường chung. Bên cạnh đó, tuỳ vào mỗi loại quy trình nghiệp vụ, các yếu tố được đánh giá có thể có những trọng số khác nhau, phụ thuộc vào người thiết kế quy trình ưu tiên yếu tố nào hơn. Chẳng hạn, đối với một quy trình A, người thiết kế xem trọng việc xử lý những ngoại lệ hơn hẳn tính linh hoạt của quy trình, dẫn đến việc trong bảng khảo sát, một điều tất yếu là điểm CES của xử lý ngoại lệ có độ ưu tiên cao hơn điểm CES của tính linh hoạt của quy trình. Điều này cũng sẽ ảnh hưởng đến điểm tổng CES của cả cuộc khảo sát.
\par
Một bài nghiên cứu về đánh giá sự hài lòng của khách hàng trong lĩnh vực ngân hàng tư nhân của ngân hàng Thương mại Hy Lạp vào năm 1999, đánh giá trên nhiều tiêu chí khác nhau:
\begin{itemize}
    \item Bộ phận nhân sự: Bao gồm các đặc trưng như: kĩ năng và kiến thức chuyên môn, tính trách nhiệm, khả năng giao tiếp và làm việc với khách hàng, sự thân thiện,…
    \item Sản phẩm: Tiêu chí này tập trung chủ yếu vào đặc điểm các sản phẩm cung cấp cho khách hàng, như độ đa dạng, khả năng bồi thường, giá cả, các dịch vụ đặc biệt,…
    \item Hình ảnh thương hiệu: Bao gồm tên tuổi, danh tiếng của ngân hàng, khả năng ứng dụng công nghệ và đáp ứng nhu cầu khách hàng trong tương lai.
    \item Dịch vụ: Liên quan tới những dịch vụ cung cấp cho khách hàng, như thời gian chờ đợi để được xử lý, độ phức tạp của các quy trình, thông tin cung cấp cho khách hàng,…
    \item Khả năng truy cập: Khả năng mở rộng mạng lưới của ngân hàng, vị trí các chi nhánh, khả năng xử lý những vấn đề có thể xảy ra với hệ thống (ATM bị lỗi).
\end{itemize}
Để tính toán được giá trị hài lòng cuối cùng (global satisfaction), trước đó cần biết được mức độ hài lòng của khách hàng trên các tiêu chí được liệt kê ở trên (partial satisfaction). Và mỗi tiêu chí lại phụ thuộc vào những đặc trưng bên trong chúng. Các tiêu chí hay đặc trưng của chúng có độ quan trọng khác nhau, đòi hỏi cần phải kết hợp các giá trị này lại với nhau thành một giá trị tổng hoà duy nhất. Tác giả đã đề xuất việc sử dụng phương pháp MUSA để xử lý vấn đề trên, và khi đối chiếu ngược trở về vấn đề xây dựng khảo sát đánh giá sự hài lòng của khách hàng đối với quy trình nghiệp vụ của chúng tôi, chúng tôi nhận thấy có sự tương đồng, bởi chúng tôi cũng khảo sát sự hài lòng trên nhiều yếu tố liên quan đến quy trình nghiệp vụ, chẳng hạn như khả năng xử lý ngoại lệ, thời gian, chi phí thực thi quy trình, sản phẩm đầu ra của quy trình. Chính vì thế, chúng tôi quyết định lựa chọn ứng dụng phương pháp MUSA vào việc tính toán mức độ hài lòng của khách hàng trên nhiều khía cạnh khác nhau của quy trình nghiệp vụ. 
\par
Multicriteria Satisfaction Analysis (MUSA) là phương pháp để đo lường và phân tích sự hài lòng của khách hàng. Phương pháp MUSA là một mô hình phân tách mức độ ưu tiên theo những nguyên tắc của phân tích hồi quy thứ tự. Phương pháp luận này sẽ đánh giá mức độ hài lòng của một tập những người dùng dựa trên giá trị của họ và những mức độ ưu tiên. Quá trình kết hợp - phân tách này sẽ được thực thi với ít khả năng xảy ra lỗi nhất. Ưu điểm của phương pháp MUSA là nó hoàn toàn xem xét chất lượng mức độ ưu tiên và đánh giá của người dùng.
\par
Mục đích chính của phương pháp MUSA là kết hợp các đánh giá độc lập thành một hàm thu thập giá trị, giả sử như độ hài lòng tổng thể của người dùng sẽ phụ thuộc vào tập n tiêu chí hay biến số đại diện cho các yếu tố khác nhau của dịch vụ. Tập các tiêu chí này được ký hiệu $X = (X_1, X_2,..., X_n)$ với mỗi tiêu chí cụ thể $i$ được đại diện bằng một biến đơn $X_i$. Bằng cách này, việc đánh giá sự hài lòng của người dùng có thể được xem như một bài toán phân tích nhiều tiêu chí.
\par
Phương pháp MUSA đánh giá độ hài lòng tổng thể và độ hài lòng ở từng yếu tố lần lượt $Y^*$ và $X_i^*$. Cần chú ý rằng phương pháp này tuân theo những nguyên tắc của phân tích hồi quy tuần tự với một số ràng buộc, sử dụng các kỹ thuật quy hoạch tuyến tính (Jacquet-Lagreze and Siskos, 1982; Siskos and Yannacopoulos, 1985; Siskos, 1985). Công thức của phân tích hồi quy tuần tự trình bày như sau:
\[ Y^* = \sum_{i=1}^{n} b_iX_i^*\]
\[ \sum_{i=1}^{n} b_i = 1\]
\par
Với $b_i$ là trọng số của tiêu chí thứ $i$ và giá trị của $Y^*$ và $X_i^*$ đã được chuẩn hoá về miền giá trị [0, 1].
Các hàm dự đoán giá trị là những kết quả quan trọng nhất của phương pháp MUSA, chúng cho thấy giá trị thực trong miền giá trị [0, 1] là giá trị mà người dùng đánh giá cho mỗi mức độ hài lòng, từng yếu tố hay tổng thể.

