\subsubsection{Tính toán giá trị chất lượng của quy trình nghiệp vụ}
Như đã đề cập, chất lượng của một quy trình nghiệp vụ bao gồm chất lượng bên ngoài và bên trong. Chúng tôi đã đề xuất phương pháp và công thức tính chất lượng bên ngoài chính là giá trị chung của bài khảo sát, và chất lượng bên trong chúng tôi giữ nguyên cách tính toán của đề tài trước.
Người thiết kế quy trình nghiệp vụ vẫn có thể thiết lập mức độ quan trọng của từng loại chất lượng, và với phương pháp MUSA, chúng tôi đề xuất công thức tính giá trị Quality của cả quy trình nghiệp vụ như sau:
\[ Q =  w_{eq} \times \text{Survey Score} + w_{iq} \times \text{Internal Quality}\]
\[w_{eq} + w_{iq} = 1\]
\par
Trong đó, $w_{eq}$, $w_{iq}$ lần lượt là trọng số của \emph{external quality} và \emph{internal quality}; Survey Score là điểm của bài khảo sát cũng như là giá trị của \emph{external quality}. Vì 2 giá trị này có chung một miền giá trị [0, 1] nên không cần phải chuẩn hóa chúng trước khi đưa vào tính toán nữa. Trọng số này có thể được quy định bởi người thiết kế quy trình nghiệp vụ, tuỳ thuộc vào độ ưu tiên của họ đối với yếu tố nào của chất lượng để đánh giá chất lượng tổng quan của quy trình nghiệp vụ.
