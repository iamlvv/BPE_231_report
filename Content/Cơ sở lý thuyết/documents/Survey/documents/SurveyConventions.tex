\subsection{Nguyên tắc thiết kế bảng khảo sát}

\subsubsection{Nguyên tắc}

Một số nguyên tắc khi thiết kế bảng khảo sát được tổng hợp bên dưới như sau:
\begin{itemize}
    \item Sử dụng từ ngữ đơn giản, quen thuộc, tránh sử dụng các từ ngữ chuyên môn, tiếng lóng, nhiều tầng lớp nghĩa, mơ hồ.
    \item Sử dụng cấu trúc câu đơn giản.
    \item Đưa ra các đáp án lựa chọn đầy đủ, toàn diện, và riêng biệt với nhau.
    \item Tránh những câu hỏi đôi, những câu hỏi phủ định hay phủ định của phủ định.
    \item Tránh những câu hỏi định hướng người dùng buộc phải chọn 
 một đáp án một cách chủ ý.
    \item Những câu hỏi mở đầu nên là những câu hỏi dễ, từ đó xây dựng được sự kết nối giữa người thiết kế câu hỏi và người trả lời câu hỏi.
    \item Các câu hỏi thuộc cùng một chủ đề nên được nhóm lại với nhau.
    \item Các câu hỏi thuộc cùng một chủ đề nên đi từ chung đến riêng.
\end{itemize}

% \subsubsection{Câu hỏi đóng và câu hỏi mở}

% Một trong số những điều mà người thiết kế câu hỏi khảo sát nên quan tâm đó là khi nào thì câu hỏi là câu hỏi mở (cho phép người dùng trả lời theo ý kiến cá nhân của họ) hoặc câu hỏi đóng (yêu cầu người dùng lựa chọn đáp án từ các lựa chọn được cung cấp sẵn). Hầu hết phần lớn các bảng khảo sát đều là những câu hỏi đóng, tuy nhiên ở một số khảo sát nghiên cứu, các câu hỏi mở vẫn có vai trò quan trọng.
% \par
% Các câu hỏi mở và đóng cũng có sự khác nhau về khả năng đo lường tính chính xác của tri thức. Các câu hỏi đóng cần phải phán đoán chính xác nhiều hơn câu hỏi mở. Để giải thích cho ý kiến này, Krosnick và Fabrigar đã khảo sát các bài kiểm tra lên học sinh cho thấy rằng các câu hỏi mở cung cấp khả năng đo lường đáng tin cậy hơn các câu hỏi đóng. Mặt khác, các câu hỏi mở lại có nguy cơ cao hơn câu hỏi đóng trong việc nhận những câu trả lời “không biết” từ những người biết đáp án chính xác nhưng họ không chắc chắn về điều đó, hoặc vì họ không thể ngay tức thì nhớ lại đáp án và không thực sự nỗ lực nhớ ra chúng.
% \par
% Vì vậy, các câu hỏi mở có lẽ chỉ phù hợp với những khảo sát mà các phản hồi “không biết” ít có khả năng xảy ra. Các câu hỏi mở cũng làm phong phú kết quả của các cuộc khảo sát mà khó thực hiện những câu hỏi đóng, hoặc các câu hỏi mở có thể được đặt theo sau những câu hỏi đóng để lấy được nhiều thông tin hơn từ người dùng.

\subsubsection{Thang điểm}

Việc chọn lựa thang điểm đánh giá cũng là một yếu tố cần được xem xét khi thiết kế bảng câu hỏi khảo sát. Likert (1932) sử dụng thang điểm 5. Osgood, Suci và Tannenbaum (1957) sử dụng thang điểm 7, hay như Thurstone (1928) lại sử dụng thang điểm 11. Không có một quy chuẩn nào cho việc sử dụng thang điểm nào trong khảo sát, dẫu vậy, chúng tôi cho rằng một số thang điểm cụ thể nên được sử dụng để tối ưu hoá dữ liệu thu được từ người dùng. 
Người dùng khi quan sát thang điểm sẽ thực hiện việc đối chiếu các mức điểm với câu trả lời đưa ra và cố gắng tìm kiếm mức điểm trùng khớp nhất có thể. Vì vậy, có một số điều kiện nhất định cần phải được thỏa mãn. 
\begin{itemize}
    \item Thứ nhất, thang đo cần bao phủ các mức độ nhiều nhất có thể, không có ngoại lệ nào.
    \item Thứ hai, các mức điểm trong thang đo cần có sự khác biệt với nhau, ý nghĩa của chúng không đan xen trùng lặp nhau.
    \item Thứ ba, người dùng và người thiết kế câu hỏi phải hiểu được ý nghĩa của mỗi mức điểm và có cách hiểu giống nhau.
\end{itemize}
\par
Nếu một vài trong những điều kiện trên không được đáp ứng có thể ảnh hưởng đến chất lượng của dữ liệu. Ví dụ, nếu câu trả lời người dùng rơi vào một trường hợp nào đó chưa được liệt kê trong thang điểm, nói cách khác, độ chi tiết của thang đo chưa cao dẫn tới chưa có đáp án thực sự phù hợp với câu trả lời của người dùng lúc đó, dẫn tới họ có thể chọn các đáp án khác nhau ở các thời điểm thực hiện khảo sát khác nhau. Ngoài ra, những đáp án có ý nghĩa tương tự nhau (chẳng hạn như “đôi khi” - “thỉnh thoảng”) sẽ khiến người dùng bối rối khi chọn lựa. Khi đó, giả sử như có nhiều người cùng thực hiện bài khảo sát này, họ có thể chọn những đáp án khác nhau vì họ có cách hiểu khác nhau đối với mỗi đáp án mặc dù câu trả lời của họ đưa ra trong đầu là giống nhau.

% \subsubsection{Thứ tự câu hỏi}
% Kết quả của cuộc khảo sát có thể bị ảnh hưởng không chỉ bởi cách diễn đạt câu hỏi, mà còn bởi bối cảnh mà câu hỏi được hỏi. Chính vì thế, trong quá trình thiết kế bộ câu hỏi, ta cũng cần quan tâm đến thứ tự các câu hỏi, nhằm tối ưu hoá độ chính xác của các phản hồi và giảm thiểu các sai sót.
% \par
% Các câu hỏi mở đầu có thể đặc biệt ảnh hưởng đến việc người dùng có sẵn sàng hoàn thành khảo sát hay không, vì nó có thể định hình tư tưởng của người dùng rằng khảo sát này nói về cái gì. Chính vì thế, các câu hỏi mở đầu thường thiết lập một kết nối mạnh mẽ đến chủ đề của cuộc khảo sát. Chúng có thể là một tập các câu hỏi đóng và dễ trả lời. Càng về cuối bảng khảo sát, người dùng dần mất đi sự hứng thú ban đầu, có một xu hướng rằng mức độ mất mát dữ liệu tăng lên, các câu hỏi lại càng ít chi tiết hơn và khả năng phân biệt các đáp án của người dùng lại càng giảm.
% \par 
% Cần phải nhóm lại với nhau những câu hỏi có chung chủ đề. Điều này hỗ trợ người dùng nhiều hơn trong việc xử lý thông tin, chẳng hạn như làm rõ ý nghĩa câu hỏi hoặc tìm kiếm trong trí nhớ dễ dàng hơn. 

