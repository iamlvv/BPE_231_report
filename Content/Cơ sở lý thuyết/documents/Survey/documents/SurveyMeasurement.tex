\subsubsection{Yếu tố khảo sát}
Một số độ đo độ hài lòng của khách hàng đối với một quy trình nghiệp vụ nói riêng và một dịch vụ, sản phẩm nói chung được áp dụng rộng rãi hiện nay có thể kể đến như: Customer Satisfaction Score (CSAT), Customer Effort Score (CES), Net Promoter Score (NPS).

\paragraph{CES}\mbox{}

CES được Corporate Executive Board (CEB) giới thiệu vào năm 2010. CES là dạng câu hỏi đánh giá sự hài lòng của người dùng bằng việc đo đạc nỗ lực của người dùng trong việc tương tác với sản phẩm hay dịch vụ, yêu cầu người dùng đánh giá nỗ lực của họ trên thang điểm từ 1 - 7, với 1 là giá trị cao nhất cho sự không đồng thuận với câu hỏi được đặt ra. 
% Theo CEB Global, điểm CES trên 2.0 được xem như là một giá trị CES tốt. Ngược lại, nếu điểm CES dưới 2.0, đây sẽ là một dấu hiệu cảnh báo cho người thiết kế cần xem xét lại những nguyên nhân dẫn đến việc người dùng không hài lòng và đưa ra những phương hướng giải quyết phù hợp. 
\par
Một số công thức tính CES đã được đưa ra, tuy nhiên mỗi công thức sẽ phù hợp với một số những thang điểm nhất định. Ở đây, chúng tôi chọn thang điểm từ 1 - 7, công thức tính thường gặp sẽ lấy số lượng người đồng ý chia cho tổng số phản hồi từ người dùng và nhân với 10 hoặc 100. Người dùng được xem như là đồng ý khi số điểm họ đánh giá cho câu hỏi này nằm trong tập giá trị \{5, 6, 7\}. Miền giá trị của điểm CES là [0, 1].
\[ \text{CES (\%)} = \frac{\text{Number of positive results}}{\text{Total of respondents}} \times 100 \]
\par
Trong đó \emph{Number of positive results} là số lượng người dùng đồng ý với câu hỏi, \emph{Total of respondents} là số lượng người dùng tham gia trả lời câu hỏi. Ví dụ, nếu chúng tôi nhận được 100 phản hồi và 70 trong số đó là phản hồi tích cực, thì điểm CES sẽ là 70\%.
% \par
% Một trong những ưu điểm của CES đó là chúng có thể cho thấy những điểm cần được cải thiện để nâng cao trải nghiệm của người dùng. Bên cạnh đó, kết quả của khảo sát CES sẽ giúp dự đoán hành vi trong tương lai. Theo như nghiên cứu của HBR, khoảng 94\% khách hàng phản hồi rằng tốn ít công sức trong việc tương tác với một doanh nghiệp sẽ mua lại sản phẩm. Và khoảng 88\% trong số đó sẽ chi nhiều tiền hơn. Nghiên cứu này cũng chỉ ra rằng CES có thể cho biết người dùng đề xuất sản phẩm tới những người khác như thế nào và họ nói về nó ra sao. Khoảng 81\% khách hàng phải tốn nhiều công sức cho biết sẽ có ý kiến tiêu cực về sản phẩm. Cho nên, ta có thể nói rằng nếu người dùng tốn ít công sức trải nghiệm sản phẩm dịch vụ hơn sẽ cho thấy thái độ tích cực của họ đối với sản phẩm, dịch vụ đó.

\paragraph{CSAT}\mbox{}

CSAT là một yếu tố đo lường mức độ hài lòng của khách hàng đối với một trải nghiệm cụ thể. CSAT thường được đo trên thang điểm từ 1 - 5, với 1 là rất không hài lòng và 5 là tuyệt đối hài lòng. Điểm CSAT nằm trong khoảng từ 75\% đến 85\% được xem như là một giá trị tốt. Ta có thể tính điểm CSAT bằng cách lấy số lượng phản hồi hài lòng chia cho tổng số phản hồi từ người dùng và nhân với 100. Người dùng được xem như là hài lòng khi họ chọn đáp án thuộc tập giá trị \{4, 5\}. Miền giá trị của điểm CSAT là [0, 1].
\[ \text{CSAT (\%)} = \frac{\text{Number of satisfied responses}}{\text{Total of respondents}} \times 100 \]
\par
Trong đó, \emph{Number of satisfied responses} là số phản hồi hài lòng từ khách hàng, \emph{Total of respondents} là tổng số phản hồi khảo sát từ người dùng. Ví dụ, nếu chúng tôi nhận được 100 phản hồi và 80 trong số đó là đánh giá hài lòng, điểm CSAT sẽ là 80\%.
% Kể từ những năm 70, CSAT đã trở thành một trong những yếu tố được sử dụng rộng rãi để đo lường mức độ hài lòng của khách hàng bởi các doanh nghiệp. Không giống như NPS, CSAT tập trung nhiều vào trải nghiệm ngay tức thì, hơn là cung cấp một đánh giá lâu dài về trải nghiệm của người dùng.

\paragraph{NPS}\mbox{}

NPS là câu hỏi được phát triển bởi Reichheld vào năm 2003, đánh giá tỉ lệ người dùng sẽ đề xuất sản phẩm hay dịch vụ cho người khác, trên thang điểm từ 0 - 10, với 0 là giá trị cho biết sản phẩm hay dịch vụ chắc chắn sẽ không được đề xuất, và 10 là giá trị đảm bảo rằng sản phẩm hay dịch vụ sẽ được đề xuất cho người khác bởi người dùng.
\par
Những phản hồi từ khảo sát NPS được phân loại thành ba nhóm chính: Promoters, Passives và Detractors. \emph{Promoters} là những người dùng có lòng trung thành cao đã đánh giá trải nghiệm của họ từ 9 đến 10. \emph{Passives} là những người trung lập và đánh giá ở mức 7 và 8. Trong khi đó, \emph{Detractors} là những người không hài lòng với sản phẩm và dịch vụ, và chỉ cho số điểm từ 0 đến 6. Điểm NPS được tính bằng tỉ lệ giữa hiệu của số người thuộc nhóm Promoters trừ đi số người thuộc nhóm Detractors trên tổng số người dùng tham gia khảo sát.
\[ NPS = \frac{\text{Promoters} - \text{Detractors}}{\text{ Total of respondents}} \times 100 \]
\par
Trong đó, \emph{Promoters} là số người thuộc nhóm \emph{Promoters}, \emph{Detractors} là số người thuộc nhóm \emph{Detractors}, \emph{Total of respondents} là tổng số phản hồi khảo sát từ người dùng.
\par
Như vậy có thể thấy, điểm NPS sẽ nằm trong khoảng giá trị từ -100\% đến 100\%, hay $NPS <= |1|$. Theo Reichheld và Markey, giá trị NPS từ -100\% đến 0 sẽ là không hài lòng, với -100\% đến -50\% là \emph{deficient}, và từ -49\% đến 0 là \emph{insufficient}. \emph{Deficient} nghĩa là chất lượng của dịch vụ được đánh giá là cực kỳ tệ. \emph{Insufficient} cho thấy người dùng đánh giá chất lượng không tốt lắm. Ngược lại, kết quả rơi vào khoảng giá trị từ 0 đến 100\% sẽ là mức độ hài lòng, với 0 đến 49\% và \emph{sufficient} và 50\% đến 100\% là \emph{excellent}. \emph{Sufficient} được hiểu rằng chất lượng của dịch vụ nhận được phản hồi khá tích cực, trong khi đó, \emph{excellent} cho thấy người dùng đánh giá rất cao.

