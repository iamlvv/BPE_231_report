\subsection{Process portfolio}

Nhìn vào thực tế, quy trình nghiệp vụ thường gắn liền với một tổ chức hoặc công ty, vì vậy việc quản lý quy trình nghiệp vụ liên quan đến một nhóm những quy trình có ảnh hưởng tới nhau. 

Tuy nhiên, không phải tất cả những bên liên quan (stakeholders) phụ trách quản lý quy trình nghiệp vụ đều có được cái nhìn tổng quát về toàn bộ quy trình trong tổ chức của họ.

Và nhu cầu xác định những quy trình tạo ra tổn thất, nguy cơ đến doanh nghiệp trở nên cấp thiết hơn, để doanh nghiệp tập trung thực hiện giám sát, thay đổi hoặc loại trừ nhóm quy trình đó.

Từ đó, mong muốn hỗ trợ những người quản lý quy trình nghiệp vụ có được góc nhìn chính xác về việc xác định những quy trình cần được cải tiến đã thúc đẩy sự hình thành của Danh mục quy trình (trong ngữ cảnh của hệ thống thì chúng ta sẽ thống nhất dùng từ Process portfolio). Process portfolio đề cập tới một lược đồ có nhiệm vụ trực quan hóa những quy trình nghiệp vụ thông qua những tiêu chí được đề ra.

Process portfolio tạo ra mức độ ưu tiên giữa những quy trình nghiệp vụ, đây là cơ sở để hiện thực những hoạt động khác (hiện thực, phân tích, đánh giá hiệu suất, tái thiết kế, quản lý,...) liên quan tới lược đồ quy trình nghiệp vụ - cũng là đối tượng chính của hệ thống BKSky 2.0.