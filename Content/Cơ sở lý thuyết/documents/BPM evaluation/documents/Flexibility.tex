\subsection{Độ linh hoạt}
Độ linh hoạt được biểu diễn bằng số lượng biến thể có thể có của một quy trình. Biến thể được tạo ra vì chúng có các khối hay các nhánh không phải thực thi. Độ linh hoạt của một quy trình được tính như sau:
\[F = \frac{\text{Num of optional tasks}}{\text{Num of total tasks}} \tag*{\cite{prereport}}\]
\par
Với \emph{Num of optional tasks} là số lượng tasks có trong khối XOR, và \emph{Num of total tasks} là tổng các tasks có trong quy trình đó.