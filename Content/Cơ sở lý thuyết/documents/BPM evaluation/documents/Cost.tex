\subsubsection{Chi phí}
Trong đề tài này, tương tự như đề tài trước, chúng tôi vẫn dựa trên \emph{Time-driven activity-based costing
model} (chi phí được tính dựa trên thời gian cần để thực hiện tất cả các task có trong quy trình đó và dựa trên chi phí đơn vị (unit cost) của khả năng cung ứng (supplying
capacity)) để tính toán chi phí của quy trình dựa vào yếu tố thời gian. Công thức tính như sau:
\[ C = UC \times T \]
\par
Trong đó C là tổng chi phí của cả quy trình, T là thời gian chu kỳ của quy trình và UC (unit cost) là chi phí đơn vị của quy trình đó. Chi phí đơn vị được tính như sau:
\[ UC = \frac{CTT}{PC}\]
\par
Trong đó CT T là tổng số thời gian mà một đơn vị (phòng ban) trong tổ chức có thể bỏ ra trong một đơn vị thời gian. P C là chi phí có thể cung cấp cho một đơn vị của tổ chức trong một đơn vị thời gian.