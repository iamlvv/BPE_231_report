\subsubsection{Chất lượng}
Chất lượng là một khái niệm trừu tượng và khó để mô hình hoá cụ thể. Ở đề tài trước, chất lượng của quy trình được đo lường dựa trên xác suất lặp lại $r$ của các khối lặp, với công thức tính như sau:
\[ Q = 1 - \frac{(r_1 + r_2 + ... + r_n)}{n}\]
trong đó $r_1$, $r_2$,..., $r_n$ là xác suất lặp lại của các vòng lặp trong quy trình, $n$ là tổng số vòng lặp có trong quy trình.
\par
Tuy nhiên, ở đề tài này, chúng tôi nhận thấy chất lượng của một quy trình không chỉ đến từ xác suất của các vòng lặp trong quy trình, mà nó còn có thể bị tác động bởi nhiều yếu tố khác. Chất lượng của một quy trình có thể được xem xét ở hai góc nhìn khác nhau: góc nhìn từ phía người sở hữu quy trình hay người thiết kế quy trình, và góc nhìn của người tham gia sử dụng, vận hành quy trình, sử dụng sản phẩm, dịch vụ của quy trình đó trên thực tế. Đây cũng có thể được xem lần lượt là \emph{internal quality} (chất lượng bên trong) và \emph{external quality} (chất lượng bên ngoài). 
\par
\emph{Internal quality} của một quy trình nghiệp vụ được xem như chất lượng quy trình trong quá trình thiết kế. Ở đây, chúng tôi cho rằng \emph{internal quality} có thể được xem như chất lượng gắn liền với số lượng và xác suất của các vòng lặp có trong quy trình, được biểu diễn và tính toán như trên. Mặt khác, \emph{external quality} lạ được đánh giá bởi sự hài lòng và trải nghiệm của những người sử dụng, tham gia vận hành hay sử dụng sản phẩm, dịch vụ của quy trình nghiệp vụ đó trong thực tế. Sự hài lòng đối với một quy trình có thể được diễn tả như mức độ mà người dùng (khách hàng) cảm nhận liệu quy trình nghiệp vụ có đáp ứng được nhu cầu của họ hay không. Sự hài lòng này không chi đơn thuần là cảm nhận trừu tượng, mà nó có thể được đo đạc với những giá trị cụ thể biểu thị cho mức độ của nó. Việc đo lường này giúp phản ánh được thái độ và kỳ vọng của khách hàng một cách hiệu quả, trực tiếp, có ý nghĩa, có thể tham khảo và so sánh. 
% Bằng cách này, sự hài lòng của khách hàng sẽ là một tiêu chuẩn cơ bản cho hiệu suất của bất kỳ sản phẩm hay dịch vụ nào, cụ thể ở đây là một quy trình nghiệp vụ trên thực tế.