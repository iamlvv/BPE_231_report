\subsection{Chất lượng}
Chất lượng là một khái niệm trừu tượng và khó để mô hình hoá cụ thể. Hệ thống ở đề tài trước, chất lượng của quy trình được đo lường dựa trên xác suất lặp lại $r$ của các khối lặp, với công thức tính như sau:
\[ Q = 1 - \frac{(r_1 + r_2 + ... + r_n)}{n} \tag*{\cite{prereport}}\]
trong đó $r_1$, $r_2$,..., $r_n$ là xác suất lặp lại của các vòng lặp trong quy trình, $n$ là tổng số vòng lặp có trong quy trình.
\par
Tuy nhiên, ở đề tài này, chúng tôi nhận thấy chất lượng của một quy trình không chỉ đến từ xác suất của các vòng lặp trong quy trình, mà nó còn có thể bị tác động bởi nhiều yếu tố khác. Chất lượng của một quy trình có thể được xem xét ở hai góc nhìn khác nhau: góc nhìn từ phía người thiết kế quy trình, và góc nhìn của người tham gia vận hành quy trình và sử dụng sản phẩm, dịch vụ của quy trình đó trên thực tế. Đây cũng có thể được xem lần lượt là \emph{internal quality} (chất lượng từ tài nguyên bên trong hệ thống) và \emph{external quality} (chất lượng từ tài nguyên bên ngoài hệ thống).
\par
\emph{Internal quality} của một quy trình nghiệp vụ trong ngữ cảnh của hệ thống ám chỉ chất lượng được tính toán thông qua mô hình quy trình mà người dùng cung cấp cho hệ thống. Ở đây, chúng tôi cho rằng \emph{internal quality} gắn liền với số lượng và xác suất của các vòng lặp có trong quy trình, được biểu diễn và tính toán như trên. Mặt khác, \emph{external quality} lại được đánh giá bởi trải nghiệm của những người tham gia vận hành hay người sử dụng sản phẩm, dịch vụ của quy trình đó trong thực tế.