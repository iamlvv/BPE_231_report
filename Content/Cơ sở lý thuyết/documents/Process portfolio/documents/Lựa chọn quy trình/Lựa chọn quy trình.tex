\subsection{Lựa chọn quy trình}
Không phải toàn bộ những quy trình đều có được sự ảnh hưởng tương đương đối với việc vận hành tổ chức, mỗi một quy trình đều có những nhiệm vụ và mục tiêu khác nhau. Vì vậy, mức độ ưu tiên khi đặt hai quy trình khác nhau lên bàn cân không ít thì nhiều sẽ khác nhau. Mục đích của quá trình lựa chọn quy trình là nhằm xác định những tiêu chí để đánh giá được hiệu suất vận hành của một quy trình nghiệp vụ, từ đó giúp người quản lý rút ra những thông tin hỗ trợ cho việc giám sát, cải thiện hay loại trừ những quy trình không hiệu quả.

Việc lựa chọn quy trình sẽ ảnh hưởng bởi nhiều yếu tố, trong số có đây là những yếu tố thường được cân nhắc:

\subsubsection{Độ quan trọng chiến lược}
    Mức độ quan trọng chiến lược liên quan tới sự ảnh hưởng hoặc giá trị của một quy trình đối với việc đạt được mục tiêu lâu dài của tổ chức. Những quy trình này có thể liên quan tới tính độc quyền thương hiệu, quy trình đóng góp vào việc tạo ra lợi thế cạnh tranh trong thị trường đối với các đối thủ khác hoặc quy trình đóng góp vào việc tạo ra giá trị cho khách hàng và đem lại nguồn lợi nhuận chính. [14]

Việc xác định được tầm quan trọng của quy trình đối với chiến lược của tổ chức sẽ là chìa khóa giúp chúng ta nhìn ra được những quy trình trọng tâm cần nhắm tới khi muốn phát triển bền vững và lâu dài.
\subsubsection{Sức khỏe}
    Health, hay gọi là Công năng của quy trình, là tiêu chí nhắm vào việc đánh giá mức độ hiệu quả và hiệu suất hoạt động của quy trình nghiệp vụ. Nó đo lường khả năng tổ chức thực hiện quy trình một cách hiệu quả, tiết kiệm và linh hoạt xử lý các trường hợp ngoại lệ nhưng cũng đồng thời đảm bảo các tiêu chuẩn về đầu ra.

Việc đánh giá tiêu chí này sẽ dễ dàng với những tổ chức hoạt động hướng quy trình - xác định nghiệp vụ dưới dạng những quy trình nghiệp vụ có thể mô hình hóa được; đây là một lợi thế cho phép doanh nghiệp thu thập được những thông tin cần thiết cho quá trình đánh giá hiệu suất. Cụ thể hơn, việc đánh giá hiệu suất quy trình nghiệp vụ sẽ được mô tả thông qua 4 tiêu chí: thời gian (time), chi phí (cost), chất lượng (quality) và độ linh hoạt (flexibility).

\begin{itemize}
      \item \textbf{Thời gian}: Thời gian là yếu tổ phổ biến trong đánh giá hiệu suất quy trình nghiệp vụ. Cụ thể hơn thì đại lượng được tính toán phổ biến là thời gian chu kỳ (cycle time) - khoảng thời gian mà quy trình nghiệp vụ bắt đầu cho đến khi kết thúc. Mục tiêu của đa phần những quy trình nghiệp vụ là giảm thời gian chu kỳ thực tế, điều này có thể hiểu là giảm thời gian chờ đợi (waiting time) - khoảng thời gian mà quy trình chờ để nhận được phản hồi từ người dùng; hoặc thời gian phục vụ (serving time) - khoảng thời gian thực sự tiêu tốn để hoàn thành tác vụ quy định trong quy trình nghiệp vụ. Trong ngữ cảnh của hệ thống hiện tại, chúng ta vẫn sẽ tập trung vào đánh giá hiệu suất của quy trình dựa vào đại lượng thời gian chu kỳ (cycle time).
      \item \textbf{Chi phí}: Khi chúng ta đề cập tới vấn đề chi phí, mục tiêu thường là giảm thiểu chi phí được sử dụng để vận hành quy trình nghiệp vụ. Chi phí này có thể là chi phí cố định (fixed cost) - chi phí không bị ảnh hưởng bởi cường độ và tần suất vận hành của quy trình nghiệp, thường là chi phí sử dụng cơ sở vật chất, chi phí bảo trì hệ thống,...; hoặc chi phí biến đổi (variable cost) - chi phí liên quan tới "lượng", ví dụ phụ thuộc vào số lượng sản phẩm, số lượng nhân viên,...Tiếp tục xem xét khía cạnh chi phí theo mô hình chi phí dựa trên hoạt động theo thời gian (Time-driven activity-based costing model); trong mô hình này chúng ta sẽ quan tâm tới chi phí đơn vị (unit cost) mà tổ chức chi trả cho quy trình nghiệp vụ để thực thi, vận hành trong một đơn vị thời gian.
      \item \textbf{Chất lượng}: Như đã trình bày ở trên, chất lượng của quy trình có thể được nhìn nhận dưới hai góc nhìn chính: từ phía mô hình quy trình (internal quality) và từ phía trải nghiệm do người tham gia quy trình chia sẻ (external quality). Chất lượng quy trình là khái niệm có phần trừu tượng, định nghĩa chất lượng có thể được tổng hợp từ nhiều yếu tố nhỏ liên quan; tuy nhiên trong ngữ cảnh hệ thống hiện tại, chúng ta sẽ tập trung phản ánh chất lượng của quy trình thông qua tính dễ hiểu của quy trình đối với người tham gia quy trình (process's participants - phân biệt với khách hàng sử dụng sản phẩm/dịch vụ cung cấp bởi đầu ra của quy trình). Trong chức năng đánh giá chất lượng quy trình, thời gian của các tác vụ trong khối lặp trong quy trình ảnh hưởng theo một trọng số lớn đến tổng thời gian của cả quy trình. Vì vậy xác suất lặp của khối lặp càng lớn thì khả năng nó tác động đến tổng thời gian của quy trình càng cao, chất lượng bên trong của quy trình sẽ bị ảnh hưởng bởi xác suất lặp của các khối lặp. Bên cạnh đó, phản hồi từ phía người tham gia quy trình nghiệp vụ cũng đóng góp vào việc đánh giá chất lượng của quy trình. Phản hồi có thể được thu thập thông qua khảo sát hoặc những lời góp ý từ phía người tham gia quy trình. Hệ thống hiện tại sẽ tập trung vào việc khai thác phản hồi từ phía người dùng để hoàn thiện hơn quá trình đánh giá chất lượng của quy trình nghiệp vụ.
      \item \textbf{Độ linh hoạt}: Độ linh hoạt của quy trình thể hiện thông qua khả năng thay đổi để thích nghi dưới những điều kiện khác nhau của quy trình nghiệp vụ. Tổ chức thường có mong muốn khiến quy trình nghiệp vụ của họ nhanh hơn, rẻ hơn và tốt hơn mà không chú ý đến yếu tố thay đổi của quy trình. Có thể dưới những hoàn cảnh khác nhau, tính ổn định vốn có của quy trình sẽ đánh mất hoàn toàn; khi đó, để so sánh hai quy trình với nhau thì quy trình có thể hoạt động ổn định trong đa số hoàn cảnh lại chiếm ưu thế hơn quy trình hoạt động tốt khi tải bình thường nhưng lại đình trệ khi tải vượt quá cao. Trong phạm vi hệ thống, định nghĩa tính linh hoạt sẽ xoay quanh Khả năng sinh ra luồng thực thi ứng với những điều kiện đầu vào khác nhau: Luồng thực thi khi này sẽ được xem xét như là một biến thể (variation) của quy trình nghiệp vụ.
\end{itemize}

Những tiêu chí được trình bày bên trên đều được đánh giá dưới dạng các đại lượng định tính/định lượng, mỗi đại lượng sẽ được thể hiện thông qua những đơn vị và thang đo khác nhau.

Nhằm có được đánh giá tổng hợp về công năng của quy trình, chúng ta cần đưa những đại lượng này về cùng một hệ quy chiếu, thông qua việc chuyển đổi đại lượng về cùng một đơn vị đo lường. Để làm được điều này, hệ thống sử dụng lý thuyết của Horkoff liên quan tới việc đo lường mức hiệu năng (performance level) của các chỉ số, mức hiệu năng này sẽ phản ánh mức độ đạt được mục tiêu của quy trình nghiệp vụ. Giá trị sẽ này được quy về khoảng giá trị từ (-1, 1).

Trước tiên, chúng ta biết rằng một đại lượng có thể là tích cực (positive), tiêu cực (negative) hoặc lưỡng cực (bidirectional); nghĩa rằng chúng ta cần tối đa hóa, tối thiểu hóa hoặc cân bằng giá trị của đại lượng đó để mức độ đạt được mục tiêu của đại lượng tăng. Để làm rõ mục tiêu của đại lượng thì chúng ta cần hiểu được lý do vì sao đại lượng đó là cần thiết. Ví dụ, chỉ số "Thời gian chu kỳ của quy trình" là cần thiết để đánh giá mục tiêu "Tối thiểu hóa thời gian chu kỳ của quy trình".

Sau khi xác định được tính chất của đại lượng, chúng ta quan tâm đến bộ 3 giá trị lần lượt là Target value - Giá trị mục tiêu, Threshold value - Giá trị ngưỡng và Worst value - Giá trị xấu nhất; trong đó:

\begin{align}
      Target{\ }value \geq Threshold{\ }value \geq Worst{\ }value
\end{align}

Bộ ba giá trị sẽ giúp chúng ta xác định được mức hiệu năng của chỉ số Current value mà người dùng nhập vào, đang ở mức độ nào trong khoảng giá trị (-1, 1).

\begin{align}
      pl(current{\ }v.) = \frac{\left| current{\ }v. -{\ }threshold{\ }v.\right|}{\left| target{\ }v. -{\ }threshold{\ }v. \right|} (current {\ }v.\geq  threshold{\ }v.)
\end{align}

hoặc

\begin{align}
      pl(current{\ }v.) = - \frac{\left| current{\ }v. -{\ }threshold{\ }v.\right|}{\left| threshold{\ }v. -{\ }worst{\ }v. \right|} (current {\ }v. <  threshold{\ }v.)
\end{align}
\subsubsection{Tính khả thi}
    Tính khả thi là tiêu chí thứ ba được cân nhắc trong process portfolio; trong phạm vi ngữ cảnh của hệ thống, tính khả thi của quy trình nghiệp vụ sẽ liên quan tới câu hỏi "Quy trình có khả năng mở rộng không?". Thay vì tập trung vào câu hỏi "Quy trình có khả năng hiện thực không?" thì khả năng mở rộng của quy trình trong tương lai sẽ ảnh hưởng nhiều hơn tới quyết định đầu tư tài nguyên và nhân lực của tổ chức. Tính khả thi thể hiện rõ ràng hơn trên con đường phát triển dài về sau; khi mà mục tiêu hiệu suất trở nên ngày càng cao thì doanh nghiệp sẽ phải nỗ lực hơn để liên tục thử nghiệm và thay đổi quy trình nghiệp vụ, quá trình này có thể kéo dài theo năm. Vì vậy để có được điểm khởi đầu tốt thì tính khả thi của quy trình nghiệp vụ là đáng được cân nhắc. [14]