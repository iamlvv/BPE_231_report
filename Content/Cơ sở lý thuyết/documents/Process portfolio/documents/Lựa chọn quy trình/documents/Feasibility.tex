Tính khả thi là tiêu chí thứ ba được cân nhắc trong process portfolio; trong phạm vi ngữ cảnh của hệ thống, tính khả thi của quy trình nghiệp vụ sẽ liên quan tới câu hỏi "Quy trình có khả năng mở rộng không?". Thay vì tập trung vào câu hỏi "Quy trình có khả năng hiện thực không?" thì khả năng mở rộng của quy trình trong tương lai sẽ ảnh hưởng nhiều hơn tới quyết định đầu tư tài nguyên và nhân lực của tổ chức. Tính khả thi thể hiện rõ ràng hơn trên con đường phát triển dài về sau; khi mà mục tiêu hiệu suất trở nên ngày càng cao thì doanh nghiệp sẽ phải nỗ lực hơn để liên tục thử nghiệm và thay đổi quy trình nghiệp vụ, quá trình này có thể kéo dài theo năm. Vì vậy để có được điểm khởi đầu tốt thì tính khả thi của quy trình nghiệp vụ là đáng được cân nhắc. [14]