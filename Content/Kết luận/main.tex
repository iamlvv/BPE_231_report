\section{Kết quả đạt được và điểm hạn chế}
Hệ thống BPSky đã trải qua hai giai đoạn phát triển chính. Trong giai đoạn trước đó, hệ thống tập trung vào cung cấp cho người dùng bộ công cụ cho phép chỉnh sửa, thiết kế quy trình nghiệp vụ dưới dạng mô hình thông qua ngôn ngữ mô hình hóa BPMN. Ngoài ra, hệ thống cũng đã thành công phát triển khung giải thuật và hiện thực tính năng cho phép người dùng đánh giá chất lượng quy trình một cách tự động. Tuy nhiên, những yếu tố như phạm vi quản lý vẫn có thể được cải tiến thêm, thay vì chỉ quản lý quy trình trong không gian làm việc của từng cá nhân thì chúng tôi mong muốn tạo một không gian chung cho phép nhiều người tham gia chia sẻ; ngoài ra tính khách quan của kết quả đánh giá chất lượng vẫn còn có phần hạn chế, hệ thống tính toán dựa trên mô hình được thiết kế nên sẽ là thiếu sót nếu quyết định chất lượng của quy trình là tốt hay chưa. Vì vậy trong giai đoạn phát triển này, hệ thống đã tập trung cải tiến 3 yếu tố chính:
\begin{enumerate}
    \item Chúng tôi đã hiện thực mở rộng không gian quản lý các dự án và các quy trình nghiệp vụ, cho phép người dùng chia sẻ không gian để mở rộng và cộng tác với nhiều thành viên khác trong hệ thống. Chủ sở hữu không gian làm việc chung giờ đây có thể quản lý thành viên cũng như xử lý được các yêu cầu của thành viên gửi tới liên quan tới phân quyền và chia sẻ đến thành viên khác ngoài không gian chung.
    \item Chúng tôi đã hiện thực và tích hợp tính năng khảo sát vào việc thiết kế mô hình quy trình nghiệp vụ, bây giờ người dùng không chỉ xây dựng mô hình quy trình nghiệp vụ mà còn có thể tham gia tùy biến khảo sát để thu thập thêm góc nhìn từ người tham gia quy trình hoặc người sử dụng sản phẩm của quy trình tạo ra.
    \item Chúng tôi đã hiện thực danh mục quy trình nghiệp vụ (process portfolio) cho mỗi không gian làm việc (workspace), danh mục quy trình cho phép người dùng thu thập cũng như điều chỉnh thông tin một cách tổng quan về toàn bộ những phiên bản quy trình nghiệp vụ đang hoạt động trong hệ thống; danh mục hỗ trợ lược đồ trực quan hóa mức độ ưu tiên giữa các quy trình dựa trên những tiêu chí mà hệ thống đề xuất.
\end{enumerate}
Qua đó, chúng tôi đã cung cấp một hệ thống quản lý quy trình nghiệp vụ mở rộng hơn về phạm vi quản lý, phạm vi cộng tác, cũng như cung cấp thêm những công cụ hỗ trợ người dùng trong việc đánh giá quy trình nghiệp vụ.
\par
Tuy nhiên, đồ án tốt nghiệp vẫn còn còn một số hạn chế. Thứ nhất, hệ thống thông báo chỉ mới có hai loại thông báo cơ bản 
là thông báo về lời mời tham gia Workspace và thông báo trạng thái của yêu cầu thay đổi quyền hạn của bản thân người dùng 
trong Workspace, và chúng tôi cho rằng cần phải mở rộng hệ thống thông báo để người dùng có thể nhận được thông báo về những 
hoạt động quan trọng khác trong Workspace cũng như trong hệ thống. Thứ hai, chúng tôi chưa xử lý hết những trường hợp 
phân quyền phức tạp hơn, liên quan đến quyền hạn trong Workspace và Project. Thứ ba, tính năng thiết kế bảng khảo sát 
vẫn còn một số hạn chế, chẳng hạn như chưa linh hoạt cho phép người dùng tạo thêm các section, hay di chuyển câu hỏi 
giữa các section. Thứ tư, tính năng quản lý kết quả bảng khảo sát chưa có thêm những chức năng như xuất file báo cáo 
theo các định dạng pdf, csv, bộ lọc chi tiết kết quả thực hiện.

\section{Hướng phát triển}
\begin{itemize}
    \item Tiếp tục mở rộng hệ thống, hướng đến việc phân loại các quy trình nghiệp vụ theo từng lĩnh vực cụ thể.
    \item Nghiên cứu, phát triển các tính năng liên quan đến việc đề xuất các mẫu quy trình thuộc lĩnh vực liên quan đến quy trình hiện tại hoặc nhu cầu của người dùng.
    \item Mở rộng thêm các tính năng đóng góp, cho phép người dùng đóng góp những quy trình, góp phần xây dựng cộng đồng phát triển quy trình nghiệp vụ.
    \item Tổ chức Workspace, Project chặt chẽ hơn, phát triển thêm những tính năng quản trị.
    \item Nghiên cứu, mở rộng thêm các hướng tiếp cận để đánh giá quy trình nghiệp vụ khách quan và chính xác hơn.
\end{itemize}